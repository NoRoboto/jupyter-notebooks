
% Default to the notebook output style

    


% Inherit from the specified cell style.




    
\documentclass[11pt]{article}

    
    
    \usepackage[T1]{fontenc}
    % Nicer default font (+ math font) than Computer Modern for most use cases
    \usepackage{mathpazo}

    % Basic figure setup, for now with no caption control since it's done
    % automatically by Pandoc (which extracts ![](path) syntax from Markdown).
    \usepackage{graphicx}
    % We will generate all images so they have a width \maxwidth. This means
    % that they will get their normal width if they fit onto the page, but
    % are scaled down if they would overflow the margins.
    \makeatletter
    \def\maxwidth{\ifdim\Gin@nat@width>\linewidth\linewidth
    \else\Gin@nat@width\fi}
    \makeatother
    \let\Oldincludegraphics\includegraphics
    % Set max figure width to be 80% of text width, for now hardcoded.
    \renewcommand{\includegraphics}[1]{\Oldincludegraphics[width=.8\maxwidth]{#1}}
    % Ensure that by default, figures have no caption (until we provide a
    % proper Figure object with a Caption API and a way to capture that
    % in the conversion process - todo).
    \usepackage{caption}
    \DeclareCaptionLabelFormat{nolabel}{}
    \captionsetup{labelformat=nolabel}

    \usepackage{adjustbox} % Used to constrain images to a maximum size 
    \usepackage{xcolor} % Allow colors to be defined
    \usepackage{enumerate} % Needed for markdown enumerations to work
    \usepackage{geometry} % Used to adjust the document margins
    \usepackage{amsmath} % Equations
    \usepackage{amssymb} % Equations
    \usepackage{textcomp} % defines textquotesingle
    % Hack from http://tex.stackexchange.com/a/47451/13684:
    \AtBeginDocument{%
        \def\PYZsq{\textquotesingle}% Upright quotes in Pygmentized code
    }
    \usepackage{upquote} % Upright quotes for verbatim code
    \usepackage{eurosym} % defines \euro
    \usepackage[mathletters]{ucs} % Extended unicode (utf-8) support
    \usepackage[utf8x]{inputenc} % Allow utf-8 characters in the tex document
    \usepackage{fancyvrb} % verbatim replacement that allows latex
    \usepackage{grffile} % extends the file name processing of package graphics 
                         % to support a larger range 
    % The hyperref package gives us a pdf with properly built
    % internal navigation ('pdf bookmarks' for the table of contents,
    % internal cross-reference links, web links for URLs, etc.)
    \usepackage{hyperref}
    \usepackage{longtable} % longtable support required by pandoc >1.10
    \usepackage{booktabs}  % table support for pandoc > 1.12.2
    \usepackage[inline]{enumitem} % IRkernel/repr support (it uses the enumerate* environment)
    \usepackage[normalem]{ulem} % ulem is needed to support strikethroughs (\sout)
                                % normalem makes italics be italics, not underlines
    

    
    
    % Colors for the hyperref package
    \definecolor{urlcolor}{rgb}{0,.145,.698}
    \definecolor{linkcolor}{rgb}{.71,0.21,0.01}
    \definecolor{citecolor}{rgb}{.12,.54,.11}

    % ANSI colors
    \definecolor{ansi-black}{HTML}{3E424D}
    \definecolor{ansi-black-intense}{HTML}{282C36}
    \definecolor{ansi-red}{HTML}{E75C58}
    \definecolor{ansi-red-intense}{HTML}{B22B31}
    \definecolor{ansi-green}{HTML}{00A250}
    \definecolor{ansi-green-intense}{HTML}{007427}
    \definecolor{ansi-yellow}{HTML}{DDB62B}
    \definecolor{ansi-yellow-intense}{HTML}{B27D12}
    \definecolor{ansi-blue}{HTML}{208FFB}
    \definecolor{ansi-blue-intense}{HTML}{0065CA}
    \definecolor{ansi-magenta}{HTML}{D160C4}
    \definecolor{ansi-magenta-intense}{HTML}{A03196}
    \definecolor{ansi-cyan}{HTML}{60C6C8}
    \definecolor{ansi-cyan-intense}{HTML}{258F8F}
    \definecolor{ansi-white}{HTML}{C5C1B4}
    \definecolor{ansi-white-intense}{HTML}{A1A6B2}

    % commands and environments needed by pandoc snippets
    % extracted from the output of `pandoc -s`
    \providecommand{\tightlist}{%
      \setlength{\itemsep}{0pt}\setlength{\parskip}{0pt}}
    \DefineVerbatimEnvironment{Highlighting}{Verbatim}{commandchars=\\\{\}}
    % Add ',fontsize=\small' for more characters per line
    \newenvironment{Shaded}{}{}
    \newcommand{\KeywordTok}[1]{\textcolor[rgb]{0.00,0.44,0.13}{\textbf{{#1}}}}
    \newcommand{\DataTypeTok}[1]{\textcolor[rgb]{0.56,0.13,0.00}{{#1}}}
    \newcommand{\DecValTok}[1]{\textcolor[rgb]{0.25,0.63,0.44}{{#1}}}
    \newcommand{\BaseNTok}[1]{\textcolor[rgb]{0.25,0.63,0.44}{{#1}}}
    \newcommand{\FloatTok}[1]{\textcolor[rgb]{0.25,0.63,0.44}{{#1}}}
    \newcommand{\CharTok}[1]{\textcolor[rgb]{0.25,0.44,0.63}{{#1}}}
    \newcommand{\StringTok}[1]{\textcolor[rgb]{0.25,0.44,0.63}{{#1}}}
    \newcommand{\CommentTok}[1]{\textcolor[rgb]{0.38,0.63,0.69}{\textit{{#1}}}}
    \newcommand{\OtherTok}[1]{\textcolor[rgb]{0.00,0.44,0.13}{{#1}}}
    \newcommand{\AlertTok}[1]{\textcolor[rgb]{1.00,0.00,0.00}{\textbf{{#1}}}}
    \newcommand{\FunctionTok}[1]{\textcolor[rgb]{0.02,0.16,0.49}{{#1}}}
    \newcommand{\RegionMarkerTok}[1]{{#1}}
    \newcommand{\ErrorTok}[1]{\textcolor[rgb]{1.00,0.00,0.00}{\textbf{{#1}}}}
    \newcommand{\NormalTok}[1]{{#1}}
    
    % Additional commands for more recent versions of Pandoc
    \newcommand{\ConstantTok}[1]{\textcolor[rgb]{0.53,0.00,0.00}{{#1}}}
    \newcommand{\SpecialCharTok}[1]{\textcolor[rgb]{0.25,0.44,0.63}{{#1}}}
    \newcommand{\VerbatimStringTok}[1]{\textcolor[rgb]{0.25,0.44,0.63}{{#1}}}
    \newcommand{\SpecialStringTok}[1]{\textcolor[rgb]{0.73,0.40,0.53}{{#1}}}
    \newcommand{\ImportTok}[1]{{#1}}
    \newcommand{\DocumentationTok}[1]{\textcolor[rgb]{0.73,0.13,0.13}{\textit{{#1}}}}
    \newcommand{\AnnotationTok}[1]{\textcolor[rgb]{0.38,0.63,0.69}{\textbf{\textit{{#1}}}}}
    \newcommand{\CommentVarTok}[1]{\textcolor[rgb]{0.38,0.63,0.69}{\textbf{\textit{{#1}}}}}
    \newcommand{\VariableTok}[1]{\textcolor[rgb]{0.10,0.09,0.49}{{#1}}}
    \newcommand{\ControlFlowTok}[1]{\textcolor[rgb]{0.00,0.44,0.13}{\textbf{{#1}}}}
    \newcommand{\OperatorTok}[1]{\textcolor[rgb]{0.40,0.40,0.40}{{#1}}}
    \newcommand{\BuiltInTok}[1]{{#1}}
    \newcommand{\ExtensionTok}[1]{{#1}}
    \newcommand{\PreprocessorTok}[1]{\textcolor[rgb]{0.74,0.48,0.00}{{#1}}}
    \newcommand{\AttributeTok}[1]{\textcolor[rgb]{0.49,0.56,0.16}{{#1}}}
    \newcommand{\InformationTok}[1]{\textcolor[rgb]{0.38,0.63,0.69}{\textbf{\textit{{#1}}}}}
    \newcommand{\WarningTok}[1]{\textcolor[rgb]{0.38,0.63,0.69}{\textbf{\textit{{#1}}}}}
    
    
    % Define a nice break command that doesn't care if a line doesn't already
    % exist.
    \def\br{\hspace*{\fill} \\* }
    % Math Jax compatability definitions
    \def\gt{>}
    \def\lt{<}
    % Document parameters
    \title{Taller1}
    
    
    

    % Pygments definitions
    
\makeatletter
\def\PY@reset{\let\PY@it=\relax \let\PY@bf=\relax%
    \let\PY@ul=\relax \let\PY@tc=\relax%
    \let\PY@bc=\relax \let\PY@ff=\relax}
\def\PY@tok#1{\csname PY@tok@#1\endcsname}
\def\PY@toks#1+{\ifx\relax#1\empty\else%
    \PY@tok{#1}\expandafter\PY@toks\fi}
\def\PY@do#1{\PY@bc{\PY@tc{\PY@ul{%
    \PY@it{\PY@bf{\PY@ff{#1}}}}}}}
\def\PY#1#2{\PY@reset\PY@toks#1+\relax+\PY@do{#2}}

\expandafter\def\csname PY@tok@w\endcsname{\def\PY@tc##1{\textcolor[rgb]{0.73,0.73,0.73}{##1}}}
\expandafter\def\csname PY@tok@c\endcsname{\let\PY@it=\textit\def\PY@tc##1{\textcolor[rgb]{0.25,0.50,0.50}{##1}}}
\expandafter\def\csname PY@tok@cp\endcsname{\def\PY@tc##1{\textcolor[rgb]{0.74,0.48,0.00}{##1}}}
\expandafter\def\csname PY@tok@k\endcsname{\let\PY@bf=\textbf\def\PY@tc##1{\textcolor[rgb]{0.00,0.50,0.00}{##1}}}
\expandafter\def\csname PY@tok@kp\endcsname{\def\PY@tc##1{\textcolor[rgb]{0.00,0.50,0.00}{##1}}}
\expandafter\def\csname PY@tok@kt\endcsname{\def\PY@tc##1{\textcolor[rgb]{0.69,0.00,0.25}{##1}}}
\expandafter\def\csname PY@tok@o\endcsname{\def\PY@tc##1{\textcolor[rgb]{0.40,0.40,0.40}{##1}}}
\expandafter\def\csname PY@tok@ow\endcsname{\let\PY@bf=\textbf\def\PY@tc##1{\textcolor[rgb]{0.67,0.13,1.00}{##1}}}
\expandafter\def\csname PY@tok@nb\endcsname{\def\PY@tc##1{\textcolor[rgb]{0.00,0.50,0.00}{##1}}}
\expandafter\def\csname PY@tok@nf\endcsname{\def\PY@tc##1{\textcolor[rgb]{0.00,0.00,1.00}{##1}}}
\expandafter\def\csname PY@tok@nc\endcsname{\let\PY@bf=\textbf\def\PY@tc##1{\textcolor[rgb]{0.00,0.00,1.00}{##1}}}
\expandafter\def\csname PY@tok@nn\endcsname{\let\PY@bf=\textbf\def\PY@tc##1{\textcolor[rgb]{0.00,0.00,1.00}{##1}}}
\expandafter\def\csname PY@tok@ne\endcsname{\let\PY@bf=\textbf\def\PY@tc##1{\textcolor[rgb]{0.82,0.25,0.23}{##1}}}
\expandafter\def\csname PY@tok@nv\endcsname{\def\PY@tc##1{\textcolor[rgb]{0.10,0.09,0.49}{##1}}}
\expandafter\def\csname PY@tok@no\endcsname{\def\PY@tc##1{\textcolor[rgb]{0.53,0.00,0.00}{##1}}}
\expandafter\def\csname PY@tok@nl\endcsname{\def\PY@tc##1{\textcolor[rgb]{0.63,0.63,0.00}{##1}}}
\expandafter\def\csname PY@tok@ni\endcsname{\let\PY@bf=\textbf\def\PY@tc##1{\textcolor[rgb]{0.60,0.60,0.60}{##1}}}
\expandafter\def\csname PY@tok@na\endcsname{\def\PY@tc##1{\textcolor[rgb]{0.49,0.56,0.16}{##1}}}
\expandafter\def\csname PY@tok@nt\endcsname{\let\PY@bf=\textbf\def\PY@tc##1{\textcolor[rgb]{0.00,0.50,0.00}{##1}}}
\expandafter\def\csname PY@tok@nd\endcsname{\def\PY@tc##1{\textcolor[rgb]{0.67,0.13,1.00}{##1}}}
\expandafter\def\csname PY@tok@s\endcsname{\def\PY@tc##1{\textcolor[rgb]{0.73,0.13,0.13}{##1}}}
\expandafter\def\csname PY@tok@sd\endcsname{\let\PY@it=\textit\def\PY@tc##1{\textcolor[rgb]{0.73,0.13,0.13}{##1}}}
\expandafter\def\csname PY@tok@si\endcsname{\let\PY@bf=\textbf\def\PY@tc##1{\textcolor[rgb]{0.73,0.40,0.53}{##1}}}
\expandafter\def\csname PY@tok@se\endcsname{\let\PY@bf=\textbf\def\PY@tc##1{\textcolor[rgb]{0.73,0.40,0.13}{##1}}}
\expandafter\def\csname PY@tok@sr\endcsname{\def\PY@tc##1{\textcolor[rgb]{0.73,0.40,0.53}{##1}}}
\expandafter\def\csname PY@tok@ss\endcsname{\def\PY@tc##1{\textcolor[rgb]{0.10,0.09,0.49}{##1}}}
\expandafter\def\csname PY@tok@sx\endcsname{\def\PY@tc##1{\textcolor[rgb]{0.00,0.50,0.00}{##1}}}
\expandafter\def\csname PY@tok@m\endcsname{\def\PY@tc##1{\textcolor[rgb]{0.40,0.40,0.40}{##1}}}
\expandafter\def\csname PY@tok@gh\endcsname{\let\PY@bf=\textbf\def\PY@tc##1{\textcolor[rgb]{0.00,0.00,0.50}{##1}}}
\expandafter\def\csname PY@tok@gu\endcsname{\let\PY@bf=\textbf\def\PY@tc##1{\textcolor[rgb]{0.50,0.00,0.50}{##1}}}
\expandafter\def\csname PY@tok@gd\endcsname{\def\PY@tc##1{\textcolor[rgb]{0.63,0.00,0.00}{##1}}}
\expandafter\def\csname PY@tok@gi\endcsname{\def\PY@tc##1{\textcolor[rgb]{0.00,0.63,0.00}{##1}}}
\expandafter\def\csname PY@tok@gr\endcsname{\def\PY@tc##1{\textcolor[rgb]{1.00,0.00,0.00}{##1}}}
\expandafter\def\csname PY@tok@ge\endcsname{\let\PY@it=\textit}
\expandafter\def\csname PY@tok@gs\endcsname{\let\PY@bf=\textbf}
\expandafter\def\csname PY@tok@gp\endcsname{\let\PY@bf=\textbf\def\PY@tc##1{\textcolor[rgb]{0.00,0.00,0.50}{##1}}}
\expandafter\def\csname PY@tok@go\endcsname{\def\PY@tc##1{\textcolor[rgb]{0.53,0.53,0.53}{##1}}}
\expandafter\def\csname PY@tok@gt\endcsname{\def\PY@tc##1{\textcolor[rgb]{0.00,0.27,0.87}{##1}}}
\expandafter\def\csname PY@tok@err\endcsname{\def\PY@bc##1{\setlength{\fboxsep}{0pt}\fcolorbox[rgb]{1.00,0.00,0.00}{1,1,1}{\strut ##1}}}
\expandafter\def\csname PY@tok@kc\endcsname{\let\PY@bf=\textbf\def\PY@tc##1{\textcolor[rgb]{0.00,0.50,0.00}{##1}}}
\expandafter\def\csname PY@tok@kd\endcsname{\let\PY@bf=\textbf\def\PY@tc##1{\textcolor[rgb]{0.00,0.50,0.00}{##1}}}
\expandafter\def\csname PY@tok@kn\endcsname{\let\PY@bf=\textbf\def\PY@tc##1{\textcolor[rgb]{0.00,0.50,0.00}{##1}}}
\expandafter\def\csname PY@tok@kr\endcsname{\let\PY@bf=\textbf\def\PY@tc##1{\textcolor[rgb]{0.00,0.50,0.00}{##1}}}
\expandafter\def\csname PY@tok@bp\endcsname{\def\PY@tc##1{\textcolor[rgb]{0.00,0.50,0.00}{##1}}}
\expandafter\def\csname PY@tok@fm\endcsname{\def\PY@tc##1{\textcolor[rgb]{0.00,0.00,1.00}{##1}}}
\expandafter\def\csname PY@tok@vc\endcsname{\def\PY@tc##1{\textcolor[rgb]{0.10,0.09,0.49}{##1}}}
\expandafter\def\csname PY@tok@vg\endcsname{\def\PY@tc##1{\textcolor[rgb]{0.10,0.09,0.49}{##1}}}
\expandafter\def\csname PY@tok@vi\endcsname{\def\PY@tc##1{\textcolor[rgb]{0.10,0.09,0.49}{##1}}}
\expandafter\def\csname PY@tok@vm\endcsname{\def\PY@tc##1{\textcolor[rgb]{0.10,0.09,0.49}{##1}}}
\expandafter\def\csname PY@tok@sa\endcsname{\def\PY@tc##1{\textcolor[rgb]{0.73,0.13,0.13}{##1}}}
\expandafter\def\csname PY@tok@sb\endcsname{\def\PY@tc##1{\textcolor[rgb]{0.73,0.13,0.13}{##1}}}
\expandafter\def\csname PY@tok@sc\endcsname{\def\PY@tc##1{\textcolor[rgb]{0.73,0.13,0.13}{##1}}}
\expandafter\def\csname PY@tok@dl\endcsname{\def\PY@tc##1{\textcolor[rgb]{0.73,0.13,0.13}{##1}}}
\expandafter\def\csname PY@tok@s2\endcsname{\def\PY@tc##1{\textcolor[rgb]{0.73,0.13,0.13}{##1}}}
\expandafter\def\csname PY@tok@sh\endcsname{\def\PY@tc##1{\textcolor[rgb]{0.73,0.13,0.13}{##1}}}
\expandafter\def\csname PY@tok@s1\endcsname{\def\PY@tc##1{\textcolor[rgb]{0.73,0.13,0.13}{##1}}}
\expandafter\def\csname PY@tok@mb\endcsname{\def\PY@tc##1{\textcolor[rgb]{0.40,0.40,0.40}{##1}}}
\expandafter\def\csname PY@tok@mf\endcsname{\def\PY@tc##1{\textcolor[rgb]{0.40,0.40,0.40}{##1}}}
\expandafter\def\csname PY@tok@mh\endcsname{\def\PY@tc##1{\textcolor[rgb]{0.40,0.40,0.40}{##1}}}
\expandafter\def\csname PY@tok@mi\endcsname{\def\PY@tc##1{\textcolor[rgb]{0.40,0.40,0.40}{##1}}}
\expandafter\def\csname PY@tok@il\endcsname{\def\PY@tc##1{\textcolor[rgb]{0.40,0.40,0.40}{##1}}}
\expandafter\def\csname PY@tok@mo\endcsname{\def\PY@tc##1{\textcolor[rgb]{0.40,0.40,0.40}{##1}}}
\expandafter\def\csname PY@tok@ch\endcsname{\let\PY@it=\textit\def\PY@tc##1{\textcolor[rgb]{0.25,0.50,0.50}{##1}}}
\expandafter\def\csname PY@tok@cm\endcsname{\let\PY@it=\textit\def\PY@tc##1{\textcolor[rgb]{0.25,0.50,0.50}{##1}}}
\expandafter\def\csname PY@tok@cpf\endcsname{\let\PY@it=\textit\def\PY@tc##1{\textcolor[rgb]{0.25,0.50,0.50}{##1}}}
\expandafter\def\csname PY@tok@c1\endcsname{\let\PY@it=\textit\def\PY@tc##1{\textcolor[rgb]{0.25,0.50,0.50}{##1}}}
\expandafter\def\csname PY@tok@cs\endcsname{\let\PY@it=\textit\def\PY@tc##1{\textcolor[rgb]{0.25,0.50,0.50}{##1}}}

\def\PYZbs{\char`\\}
\def\PYZus{\char`\_}
\def\PYZob{\char`\{}
\def\PYZcb{\char`\}}
\def\PYZca{\char`\^}
\def\PYZam{\char`\&}
\def\PYZlt{\char`\<}
\def\PYZgt{\char`\>}
\def\PYZsh{\char`\#}
\def\PYZpc{\char`\%}
\def\PYZdl{\char`\$}
\def\PYZhy{\char`\-}
\def\PYZsq{\char`\'}
\def\PYZdq{\char`\"}
\def\PYZti{\char`\~}
% for compatibility with earlier versions
\def\PYZat{@}
\def\PYZlb{[}
\def\PYZrb{]}
\makeatother


    % Exact colors from NB
    \definecolor{incolor}{rgb}{0.0, 0.0, 0.5}
    \definecolor{outcolor}{rgb}{0.545, 0.0, 0.0}



    
    % Prevent overflowing lines due to hard-to-break entities
    \sloppy 
    % Setup hyperref package
    \hypersetup{
      breaklinks=true,  % so long urls are correctly broken across lines
      colorlinks=true,
      urlcolor=urlcolor,
      linkcolor=linkcolor,
      citecolor=citecolor,
      }
    % Slightly bigger margins than the latex defaults
    
    \geometry{verbose,tmargin=1in,bmargin=1in,lmargin=1in,rmargin=1in}
    
    

    \begin{document}
    
    
    \maketitle
    
    

    
    \section{TALLER DE OPTIMIZACIÓN}\label{taller-de-optimizaciuxf3n}

\subsubsection{Presentado por: Daniel Diaz - Alejandro
Suarez}\label{presentado-por-daniel-diaz---alejandro-suarez}

    1- Método gráfico.

Dada la condición de k de ser irrestricta para los valores aproximados
de 0.5, se considera entonces la región factible por debajo de la recta
Z = 8, se observa que x1 y x2 en dicho punto es óptimo (intercepto de
pendiente m = -0.5).

Dónde:

\begin{itemize}
\tightlist
\item
  Si K=1 -\textgreater{} X1=2, X2=3, Z=8.
\item
  Si K=2 -\textgreater{} X1=2, X2=3, Z=8.
\item
  Si K=n -\textgreater{} X1=2, X2=3, Z=8.
\end{itemize}

    2- A) Método gráfico:

\begin{enumerate}
\def\labelenumi{\Alph{enumi})}
\setcounter{enumi}{1}
\tightlist
\item
  Reformulación:
\end{enumerate}

\begin{itemize}
\tightlist
\item
  F.O =\textgreater{} Z = -x1' + x1'' + 4x2' - 12
\item
  R1 =\textgreater{} -3x1' + 3x1'' + x2' ≤ 9
\item
  R2 =\textgreater{} x1' - x1'' + 2x2' ≤ 9
\end{itemize}

\begin{enumerate}
\def\labelenumi{\Alph{enumi})}
\setcounter{enumi}{2}
\tightlist
\item
  Resolver por método simplex:
\end{enumerate}

    \begin{Verbatim}[commandchars=\\\{\}]
{\color{incolor}In [{\color{incolor}2}]:} \PY{k+kn}{from} \PY{n+nn}{scipy}\PY{n+nn}{.}\PY{n+nn}{optimize} \PY{k}{import} \PY{n}{linprog}
        \PY{k+kn}{import} \PY{n+nn}{numpy} \PY{k}{as} \PY{n+nn}{np}
        
        \PY{c+c1}{\PYZsh{} objective fuction}
        \PY{n}{z} \PY{o}{=} \PY{n}{np}\PY{o}{.}\PY{n}{array}\PY{p}{(}\PY{p}{[}\PY{o}{\PYZhy{}}\PY{l+m+mi}{1}\PY{p}{,} \PY{l+m+mi}{4}\PY{p}{]}\PY{p}{)}
        
        \PY{c+c1}{\PYZsh{} Constraints}
        \PY{n}{C} \PY{o}{=} \PY{n}{np}\PY{o}{.}\PY{n}{array}\PY{p}{(}\PY{p}{[}
            \PY{p}{[}\PY{o}{\PYZhy{}}\PY{l+m+mi}{3}\PY{p}{,} \PY{l+m+mi}{1}\PY{p}{]}\PY{p}{,}  \PY{c+c1}{\PYZsh{} C1}
            \PY{p}{[}\PY{l+m+mi}{1}\PY{p}{,} \PY{l+m+mi}{2}\PY{p}{]}\PY{p}{,}
            \PY{p}{[}\PY{l+m+mi}{0}\PY{p}{,} \PY{o}{\PYZhy{}}\PY{l+m+mi}{1}\PY{p}{]}  \PY{c+c1}{\PYZsh{} C2}
        \PY{p}{]}\PY{p}{)}
        
        \PY{c+c1}{\PYZsh{} upper\PYZus{}bound}
        \PY{n}{b} \PY{o}{=} \PY{n}{np}\PY{o}{.}\PY{n}{array}\PY{p}{(}\PY{p}{[}\PY{l+m+mi}{6}\PY{p}{,} \PY{l+m+mi}{3}\PY{p}{,} \PY{l+m+mi}{3}\PY{p}{]}\PY{p}{)}
        
        \PY{c+c1}{\PYZsh{} bounds \PYZgt{}= 0 for each variable}
        \PY{n}{x1} \PY{o}{=} \PY{p}{(}\PY{k+kc}{None}\PY{p}{,} \PY{k+kc}{None}\PY{p}{)}
        \PY{n}{x2} \PY{o}{=} \PY{p}{(}\PY{k+kc}{None}\PY{p}{,} \PY{k+kc}{None}\PY{p}{)}
        
        \PY{c+c1}{\PYZsh{} This LP problem has a maximize objective function and linprog supports minimize problems. To solve it:}
        \PY{n}{sol} \PY{o}{=} \PY{n}{linprog}\PY{p}{(}\PY{o}{\PYZhy{}}\PY{n}{z}\PY{p}{,} \PY{n}{A\PYZus{}ub}\PY{o}{=}\PY{n}{C}\PY{p}{,} \PY{n}{b\PYZus{}ub}\PY{o}{=}\PY{n}{b}\PY{p}{,} \PY{n}{bounds}\PY{o}{=}\PY{p}{(}\PY{n}{x1}\PY{p}{,} \PY{n}{x2}\PY{p}{)}\PY{p}{,} \PY{n}{method}\PY{o}{=}\PY{l+s+s1}{\PYZsq{}}\PY{l+s+s1}{simplex}\PY{l+s+s1}{\PYZsq{}}\PY{p}{)}
        
        \PY{n+nb}{print}\PY{p}{(}\PY{n}{sol}\PY{p}{)}
        
        \PY{n+nb}{print}\PY{p}{(}\PY{l+s+s2}{\PYZdq{}}\PY{l+s+s2}{x1 = }\PY{l+s+si}{\PYZob{}\PYZcb{}}\PY{l+s+s2}{, x2 = }\PY{l+s+si}{\PYZob{}\PYZcb{}}\PY{l+s+s2}{, z = }\PY{l+s+si}{\PYZob{}\PYZcb{}}\PY{l+s+s2}{\PYZdq{}}\PY{o}{.}\PY{n}{format}\PY{p}{(}\PY{n}{sol}\PY{o}{.}\PY{n}{x}\PY{p}{[}\PY{l+m+mi}{0}\PY{p}{]}\PY{p}{,} \PY{n}{sol}\PY{o}{.}\PY{n}{x}\PY{p}{[}\PY{l+m+mi}{1}\PY{p}{]}\PY{p}{,} \PY{n}{sol}\PY{o}{.}\PY{n}{fun}\PY{o}{*}\PY{o}{\PYZhy{}}\PY{l+m+mi}{1}\PY{p}{)}\PY{p}{)}
\end{Verbatim}


    \begin{Verbatim}[commandchars=\\\{\}]
     fun: -9.857142857142858
 message: 'Optimization terminated successfully.'
     nit: 2
   slack: array([0.        , 0.        , 5.14285714])
  status: 0
 success: True
       x: array([-1.28571429,  2.14285714])
x1 = -1.2857142857142858, x2 = 2.1428571428571432, z = 9.857142857142858

    \end{Verbatim}

    3- Utilice la idea fundamental del método Simplex para identificar los
números que faltan en la tabla:

\begin{longtable}[]{@{}lllllllc@{}}
\toprule
& x1 & x2 & x3 & x4 & x5 & x6 & RHS\tabularnewline
\midrule
\endhead
z & 0 & 7 & 0 & 2 & 0 & 2 & 6\tabularnewline
x5 & 0 & 4 & 0 & 1 & 1 & 2 & 7\tabularnewline
x3 & 0 & 4 & 1 & -2 & 0 & 4 & 0\tabularnewline
x1 & 1 & 0 & 0 & 1 & 0 & -1 & 1\tabularnewline
\bottomrule
\end{longtable}

Usando:

\begin{longtable}[]{@{}lllc@{}}
\toprule
Z & x0 & XN & RHS\tabularnewline
\midrule
\endhead
1 & 0 & cB\^{}(-1) * B \^{} -1 * N - cN \^{} t & cB\^{}(-1) * B \^{}
-1\tabularnewline
0 & I & B \^{} -1 * N & B \^{} -1 * b\tabularnewline
\bottomrule
\end{longtable}

Dónde:

\begin{itemize}
\tightlist
\item
  B\^{}(-1) * N:
\end{itemize}

\begin{longtable}[]{@{}lll@{}}
\toprule
1 & 1 & 2\tabularnewline
-2 & 0 & 4\tabularnewline
1 & 0 & -1\tabularnewline
\bottomrule
\end{longtable}

\begin{itemize}
\tightlist
\item
  N (Coeficientes de las restricciones):
\end{itemize}

\begin{longtable}[]{@{}lll@{}}
\toprule
2 & 2 & 1/2\tabularnewline
-4 & -2 & -3/2\tabularnewline
1 & 2 & 1/2\tabularnewline
\bottomrule
\end{longtable}

\begin{itemize}
\tightlist
\item
  Cb \^{} t: Coeficientes de las variables básicas.
\end{itemize}

\begin{longtable}[]{@{}lll@{}}
\toprule
X5 & X3 & X1\tabularnewline
\midrule
\endhead
0 & 2 & 6\tabularnewline
\bottomrule
\end{longtable}

    4- Resolver por simplex:

\begin{enumerate}
\def\labelenumi{\Alph{enumi})}
\item
\end{enumerate}

    \begin{Verbatim}[commandchars=\\\{\}]
{\color{incolor}In [{\color{incolor}6}]:} \PY{k+kn}{from} \PY{n+nn}{scipy}\PY{n+nn}{.}\PY{n+nn}{optimize} \PY{k}{import} \PY{n}{linprog}
        \PY{k+kn}{import} \PY{n+nn}{numpy} \PY{k}{as} \PY{n+nn}{np}
        
        
        \PY{c+c1}{\PYZsh{}objective fuction}
        \PY{n}{z} \PY{o}{=} \PY{n}{np}\PY{o}{.}\PY{n}{array}\PY{p}{(}\PY{p}{[}\PY{l+m+mi}{2}\PY{p}{,}\PY{o}{\PYZhy{}}\PY{l+m+mi}{1}\PY{p}{,}\PY{l+m+mi}{1}\PY{p}{]}\PY{p}{)}
        
        \PY{c+c1}{\PYZsh{}Constraints}
        \PY{n}{C} \PY{o}{=} \PY{n}{np}\PY{o}{.}\PY{n}{array}\PY{p}{(}\PY{p}{[}
            \PY{p}{[}\PY{l+m+mi}{2}\PY{p}{,}\PY{l+m+mi}{1}\PY{p}{,}\PY{o}{\PYZhy{}}\PY{l+m+mi}{2}\PY{p}{]}\PY{p}{,}          \PY{c+c1}{\PYZsh{}C1}
            \PY{p}{[}\PY{o}{\PYZhy{}}\PY{l+m+mi}{4}\PY{p}{,}\PY{l+m+mi}{1}\PY{p}{,}\PY{o}{\PYZhy{}}\PY{l+m+mi}{2}\PY{p}{]}\PY{p}{,}
            \PY{p}{[}\PY{o}{\PYZhy{}}\PY{l+m+mi}{2}\PY{p}{,}\PY{o}{\PYZhy{}}\PY{l+m+mi}{3}\PY{p}{,}\PY{l+m+mi}{1}\PY{p}{]}           \PY{c+c1}{\PYZsh{}C2}
        \PY{p}{]}\PY{p}{)}
        
        \PY{c+c1}{\PYZsh{}upper\PYZus{}bound}
        \PY{n}{b} \PY{o}{=} \PY{n}{np}\PY{o}{.}\PY{n}{array}\PY{p}{(}\PY{p}{[}\PY{l+m+mi}{8}\PY{p}{,}\PY{o}{\PYZhy{}}\PY{l+m+mi}{2}\PY{p}{,}\PY{o}{\PYZhy{}}\PY{l+m+mi}{4}\PY{p}{]}\PY{p}{)}
        
        \PY{c+c1}{\PYZsh{}bounds \PYZgt{}= 0 for each variable}
        \PY{n}{x1} \PY{o}{=} \PY{p}{(}\PY{l+m+mi}{0}\PY{p}{,} \PY{k+kc}{None}\PY{p}{)}
        \PY{n}{x2} \PY{o}{=} \PY{p}{(}\PY{l+m+mi}{0}\PY{p}{,} \PY{k+kc}{None}\PY{p}{)}
        \PY{n}{x3} \PY{o}{=} \PY{p}{(}\PY{l+m+mi}{0}\PY{p}{,} \PY{k+kc}{None}\PY{p}{)}
        
        \PY{c+c1}{\PYZsh{}This LP problem has a maximize objective function and linprog supports minimize problems. To solve it:}
        \PY{n}{sol} \PY{o}{=} \PY{n}{linprog}\PY{p}{(}\PY{o}{\PYZhy{}}\PY{n}{z}\PY{p}{,} \PY{n}{A\PYZus{}ub} \PY{o}{=} \PY{n}{C}\PY{p}{,} \PY{n}{b\PYZus{}ub} \PY{o}{=} \PY{n}{b}\PY{p}{,} \PY{n}{bounds} \PY{o}{=} \PY{p}{(}\PY{n}{x1}\PY{p}{,} \PY{n}{x2}\PY{p}{,} \PY{n}{x3}\PY{p}{)}\PY{p}{,} \PY{n}{method}\PY{o}{=}\PY{l+s+s1}{\PYZsq{}}\PY{l+s+s1}{simplex}\PY{l+s+s1}{\PYZsq{}}\PY{p}{)}
        
        \PY{n+nb}{print}\PY{p}{(}\PY{n}{sol}\PY{p}{)}
        
        \PY{n+nb}{print}\PY{p}{(}\PY{l+s+s2}{\PYZdq{}}\PY{l+s+s2}{x1 = }\PY{l+s+si}{\PYZob{}\PYZcb{}}\PY{l+s+s2}{, x2 = }\PY{l+s+si}{\PYZob{}\PYZcb{}}\PY{l+s+s2}{, z = }\PY{l+s+si}{\PYZob{}\PYZcb{}}\PY{l+s+s2}{\PYZdq{}}\PY{o}{.}\PY{n}{format}\PY{p}{(}\PY{n}{sol}\PY{o}{.}\PY{n}{x}\PY{p}{[}\PY{l+m+mi}{0}\PY{p}{]}\PY{p}{,} \PY{n}{sol}\PY{o}{.}\PY{n}{x}\PY{p}{[}\PY{l+m+mi}{1}\PY{p}{]}\PY{p}{,} \PY{n}{sol}\PY{o}{.}\PY{n}{fun}\PY{o}{*}\PY{o}{\PYZhy{}}\PY{l+m+mi}{1}\PY{p}{)}\PY{p}{)}
\end{Verbatim}


    \begin{Verbatim}[commandchars=\\\{\}]
     fun: -4.0
 message: 'Optimization failed. The problem appears to be unbounded.'
     nit: 3
   slack: array([4., 6., 0.])
  status: 3
 success: False
       x: array([2., 0., 0.])
x1 = 2.0, x2 = 0.0, z = 4.0

    \end{Verbatim}

    El problema no tiene solución.

    \begin{enumerate}
\def\labelenumi{\Alph{enumi})}
\setcounter{enumi}{2}
\item
\end{enumerate}

    \begin{Verbatim}[commandchars=\\\{\}]
{\color{incolor}In [{\color{incolor}4}]:} \PY{k+kn}{from} \PY{n+nn}{scipy}\PY{n+nn}{.}\PY{n+nn}{optimize} \PY{k}{import} \PY{n}{linprog}
        \PY{k+kn}{import} \PY{n+nn}{numpy} \PY{k}{as} \PY{n+nn}{np}
        
        
        \PY{c+c1}{\PYZsh{}objective fuction}
        \PY{n}{z} \PY{o}{=} \PY{n}{np}\PY{o}{.}\PY{n}{array}\PY{p}{(}\PY{p}{[}\PY{l+m+mi}{1}\PY{p}{,}\PY{l+m+mi}{3}\PY{p}{,}\PY{o}{\PYZhy{}}\PY{l+m+mi}{1}\PY{p}{]}\PY{p}{)}
        
        \PY{c+c1}{\PYZsh{}Constraints}
        \PY{n}{C} \PY{o}{=} \PY{n}{np}\PY{o}{.}\PY{n}{array}\PY{p}{(}\PY{p}{[}
            \PY{p}{[}\PY{o}{\PYZhy{}}\PY{l+m+mi}{2}\PY{p}{,}\PY{o}{\PYZhy{}}\PY{l+m+mi}{1}\PY{p}{,}\PY{o}{\PYZhy{}}\PY{l+m+mi}{1}\PY{p}{]}\PY{p}{,}          \PY{c+c1}{\PYZsh{}C1}
            \PY{p}{[}\PY{l+m+mi}{1}\PY{p}{,}\PY{o}{\PYZhy{}}\PY{l+m+mi}{1}\PY{p}{,}\PY{l+m+mi}{0}\PY{p}{]}\PY{p}{,}
            \PY{p}{[}\PY{o}{\PYZhy{}}\PY{l+m+mi}{1}\PY{p}{,}\PY{l+m+mi}{5}\PY{p}{,}\PY{l+m+mi}{1}\PY{p}{]}           \PY{c+c1}{\PYZsh{}C2}
        \PY{p}{]}\PY{p}{)}
        
        \PY{c+c1}{\PYZsh{}upper\PYZus{}bound}
        \PY{n}{b} \PY{o}{=} \PY{n}{np}\PY{o}{.}\PY{n}{array}\PY{p}{(}\PY{p}{[}\PY{o}{\PYZhy{}}\PY{l+m+mi}{3}\PY{p}{,}\PY{o}{\PYZhy{}}\PY{l+m+mi}{2}\PY{p}{,}\PY{l+m+mi}{4}\PY{p}{]}\PY{p}{)}
        
        \PY{c+c1}{\PYZsh{}bounds \PYZgt{}= 0 for each variable}
        \PY{n}{x1} \PY{o}{=} \PY{p}{(}\PY{l+m+mi}{0}\PY{p}{,} \PY{k+kc}{None}\PY{p}{)}
        \PY{n}{x2} \PY{o}{=} \PY{p}{(}\PY{l+m+mi}{0}\PY{p}{,} \PY{k+kc}{None}\PY{p}{)}
        \PY{n}{x3} \PY{o}{=} \PY{p}{(}\PY{l+m+mi}{0}\PY{p}{,} \PY{k+kc}{None}\PY{p}{)}
        
        \PY{c+c1}{\PYZsh{}This LP problem has a maximize objective function and linprog supports minimize problems. To solve it:}
        \PY{n}{sol} \PY{o}{=} \PY{n}{linprog}\PY{p}{(}\PY{n}{z}\PY{p}{,} \PY{n}{A\PYZus{}ub} \PY{o}{=} \PY{n}{C}\PY{p}{,} \PY{n}{b\PYZus{}ub} \PY{o}{=} \PY{n}{b}\PY{p}{,} \PY{n}{bounds} \PY{o}{=} \PY{p}{(}\PY{n}{x1}\PY{p}{,} \PY{n}{x2}\PY{p}{,} \PY{n}{x3}\PY{p}{)}\PY{p}{,} \PY{n}{method}\PY{o}{=}\PY{l+s+s1}{\PYZsq{}}\PY{l+s+s1}{simplex}\PY{l+s+s1}{\PYZsq{}}\PY{p}{)}
        
        \PY{n+nb}{print}\PY{p}{(}\PY{n}{sol}\PY{p}{)}
        
        \PY{n+nb}{print}\PY{p}{(}\PY{l+s+s2}{\PYZdq{}}\PY{l+s+s2}{x1 = }\PY{l+s+si}{\PYZob{}\PYZcb{}}\PY{l+s+s2}{, x2 = }\PY{l+s+si}{\PYZob{}\PYZcb{}}\PY{l+s+s2}{, z = }\PY{l+s+si}{\PYZob{}\PYZcb{}}\PY{l+s+s2}{\PYZdq{}}\PY{o}{.}\PY{n}{format}\PY{p}{(}\PY{n}{sol}\PY{o}{.}\PY{n}{x}\PY{p}{[}\PY{l+m+mi}{0}\PY{p}{]}\PY{p}{,} \PY{n}{sol}\PY{o}{.}\PY{n}{x}\PY{p}{[}\PY{l+m+mi}{1}\PY{p}{]}\PY{p}{,} \PY{n}{sol}\PY{o}{.}\PY{n}{fun}\PY{o}{*}\PY{o}{\PYZhy{}}\PY{l+m+mi}{1}\PY{p}{)}\PY{p}{)}
\end{Verbatim}


    \begin{Verbatim}[commandchars=\\\{\}]
     fun: 1.75
 message: 'Optimization failed. Unable to find a feasible starting point.'
     nit: 3
  status: 2
 success: False
       x: nan

    \end{Verbatim}

    \begin{Verbatim}[commandchars=\\\{\}]

        ---------------------------------------------------------------------------

        TypeError                                 Traceback (most recent call last)

        <ipython-input-4-6a84ebb04b98> in <module>()
         26 print(sol)
         27 
    ---> 28 print("x1 = \{\}, x2 = \{\}, z = \{\}".format(sol.x[0], sol.x[1], sol.fun*-1))
    

        TypeError: 'float' object is not subscriptable

    \end{Verbatim}

    El problema no tiene solución

    \begin{enumerate}
\def\labelenumi{\Alph{enumi})}
\setcounter{enumi}{3}
\item
\end{enumerate}

    \begin{Verbatim}[commandchars=\\\{\}]
{\color{incolor}In [{\color{incolor}5}]:} \PY{k+kn}{from} \PY{n+nn}{scipy}\PY{n+nn}{.}\PY{n+nn}{optimize} \PY{k}{import} \PY{n}{linprog}
        \PY{k+kn}{import} \PY{n+nn}{numpy} \PY{k}{as} \PY{n+nn}{np}
        
        \PY{c+c1}{\PYZsh{}objective fuction}
        \PY{n}{z} \PY{o}{=} \PY{n}{np}\PY{o}{.}\PY{n}{array}\PY{p}{(}\PY{p}{[}\PY{l+m+mi}{4}\PY{p}{,}\PY{l+m+mi}{5}\PY{p}{,}\PY{o}{\PYZhy{}}\PY{l+m+mi}{3}\PY{p}{]}\PY{p}{)}
        
        \PY{c+c1}{\PYZsh{}Constraints}
        \PY{n}{C} \PY{o}{=} \PY{n}{np}\PY{o}{.}\PY{n}{array}\PY{p}{(}\PY{p}{[}
            \PY{p}{[}\PY{l+m+mi}{1}\PY{p}{,}\PY{l+m+mi}{1}\PY{p}{,}\PY{l+m+mi}{1}\PY{p}{]}\PY{p}{,}          
            \PY{p}{[}\PY{o}{\PYZhy{}}\PY{l+m+mi}{1}\PY{p}{,}\PY{o}{\PYZhy{}}\PY{l+m+mi}{1}\PY{p}{,}\PY{o}{\PYZhy{}}\PY{l+m+mi}{1}\PY{p}{]}\PY{p}{,}
            \PY{p}{[}\PY{o}{\PYZhy{}}\PY{l+m+mi}{1}\PY{p}{,}\PY{l+m+mi}{1}\PY{p}{,}\PY{l+m+mi}{0}\PY{p}{]}\PY{p}{,}
            \PY{p}{[}\PY{l+m+mi}{1}\PY{p}{,}\PY{l+m+mi}{3}\PY{p}{,}\PY{l+m+mi}{1}\PY{p}{]}          
        \PY{p}{]}\PY{p}{)}
        
        \PY{c+c1}{\PYZsh{}upper\PYZus{}bound}
        \PY{n}{b} \PY{o}{=} \PY{n}{np}\PY{o}{.}\PY{n}{array}\PY{p}{(}\PY{p}{[}\PY{o}{\PYZhy{}}\PY{l+m+mi}{10}\PY{p}{,}\PY{o}{\PYZhy{}}\PY{l+m+mi}{10}\PY{p}{,}\PY{o}{\PYZhy{}}\PY{l+m+mi}{1}\PY{p}{,}\PY{l+m+mi}{14}\PY{p}{]}\PY{p}{)}
        
        \PY{c+c1}{\PYZsh{}bounds \PYZgt{}= 0 for each variable}
        \PY{n}{x1} \PY{o}{=} \PY{p}{(}\PY{l+m+mi}{0}\PY{p}{,} \PY{k+kc}{None}\PY{p}{)}
        \PY{n}{x2} \PY{o}{=} \PY{p}{(}\PY{l+m+mi}{0}\PY{p}{,} \PY{k+kc}{None}\PY{p}{)}
        \PY{n}{x3} \PY{o}{=} \PY{p}{(}\PY{l+m+mi}{0}\PY{p}{,} \PY{k+kc}{None}\PY{p}{)}
        
        \PY{c+c1}{\PYZsh{}This LP problem has a maximize objective function and linprog supports minimize problems. To solve it:}
        \PY{n}{sol} \PY{o}{=} \PY{n}{linprog}\PY{p}{(}\PY{o}{\PYZhy{}}\PY{n}{z}\PY{p}{,} \PY{n}{A\PYZus{}ub} \PY{o}{=} \PY{n}{C}\PY{p}{,} \PY{n}{b\PYZus{}ub} \PY{o}{=} \PY{n}{b}\PY{p}{,} \PY{n}{bounds} \PY{o}{=} \PY{p}{(}\PY{n}{x1}\PY{p}{,} \PY{n}{x2}\PY{p}{,} \PY{n}{x3}\PY{p}{)}\PY{p}{,} \PY{n}{method}\PY{o}{=}\PY{l+s+s1}{\PYZsq{}}\PY{l+s+s1}{simplex}\PY{l+s+s1}{\PYZsq{}}\PY{p}{)}
        
        \PY{n+nb}{print}\PY{p}{(}\PY{n}{sol}\PY{p}{)}
        
        \PY{n+nb}{print}\PY{p}{(}\PY{l+s+s2}{\PYZdq{}}\PY{l+s+s2}{x1 = }\PY{l+s+si}{\PYZob{}\PYZcb{}}\PY{l+s+s2}{, x2 = }\PY{l+s+si}{\PYZob{}\PYZcb{}}\PY{l+s+s2}{, z = }\PY{l+s+si}{\PYZob{}\PYZcb{}}\PY{l+s+s2}{\PYZdq{}}\PY{o}{.}\PY{n}{format}\PY{p}{(}\PY{n}{sol}\PY{o}{.}\PY{n}{x}\PY{p}{[}\PY{l+m+mi}{0}\PY{p}{]}\PY{p}{,} \PY{n}{sol}\PY{o}{.}\PY{n}{x}\PY{p}{[}\PY{l+m+mi}{1}\PY{p}{]}\PY{p}{,} \PY{n}{sol}\PY{o}{.}\PY{n}{fun}\PY{o}{*}\PY{o}{\PYZhy{}}\PY{l+m+mi}{1}\PY{p}{)}\PY{p}{)}
\end{Verbatim}


    \begin{Verbatim}[commandchars=\\\{\}]
     fun: 20.0
 message: 'Optimization failed. Unable to find a feasible starting point.'
     nit: 1
  status: 2
 success: False
       x: nan

    \end{Verbatim}

    \begin{Verbatim}[commandchars=\\\{\}]

        ---------------------------------------------------------------------------

        TypeError                                 Traceback (most recent call last)

        <ipython-input-5-b2f89c27d132> in <module>()
         26 print(sol)
         27 
    ---> 28 print("x1 = \{\}, x2 = \{\}, z = \{\}".format(sol.x[0], sol.x[1], sol.fun*-1))
    

        TypeError: 'float' object is not subscriptable

    \end{Verbatim}

    El problema no tiene solución


    % Add a bibliography block to the postdoc
    
    
    
    \end{document}


% Default to the notebook output style

    


% Inherit from the specified cell style.




    
\documentclass[11pt]{article}

    
    
    \usepackage[T1]{fontenc}
    % Nicer default font (+ math font) than Computer Modern for most use cases
    \usepackage{mathpazo}

    % Basic figure setup, for now with no caption control since it's done
    % automatically by Pandoc (which extracts ![](path) syntax from Markdown).
    \usepackage{graphicx}
    % We will generate all images so they have a width \maxwidth. This means
    % that they will get their normal width if they fit onto the page, but
    % are scaled down if they would overflow the margins.
    \makeatletter
    \def\maxwidth{\ifdim\Gin@nat@width>\linewidth\linewidth
    \else\Gin@nat@width\fi}
    \makeatother
    \let\Oldincludegraphics\includegraphics
    % Set max figure width to be 80% of text width, for now hardcoded.
    \renewcommand{\includegraphics}[1]{\Oldincludegraphics[width=.8\maxwidth]{#1}}
    % Ensure that by default, figures have no caption (until we provide a
    % proper Figure object with a Caption API and a way to capture that
    % in the conversion process - todo).
    \usepackage{caption}
    \DeclareCaptionLabelFormat{nolabel}{}
    \captionsetup{labelformat=nolabel}

    \usepackage{adjustbox} % Used to constrain images to a maximum size 
    \usepackage{xcolor} % Allow colors to be defined
    \usepackage{enumerate} % Needed for markdown enumerations to work
    \usepackage{geometry} % Used to adjust the document margins
    \usepackage{amsmath} % Equations
    \usepackage{amssymb} % Equations
    \usepackage{textcomp} % defines textquotesingle
    % Hack from http://tex.stackexchange.com/a/47451/13684:
    \AtBeginDocument{%
        \def\PYZsq{\textquotesingle}% Upright quotes in Pygmentized code
    }
    \usepackage{upquote} % Upright quotes for verbatim code
    \usepackage{eurosym} % defines \euro
    \usepackage[mathletters]{ucs} % Extended unicode (utf-8) support
    \usepackage[utf8x]{inputenc} % Allow utf-8 characters in the tex document
    \usepackage{fancyvrb} % verbatim replacement that allows latex
    \usepackage{grffile} % extends the file name processing of package graphics 
                         % to support a larger range 
    % The hyperref package gives us a pdf with properly built
    % internal navigation ('pdf bookmarks' for the table of contents,
    % internal cross-reference links, web links for URLs, etc.)
    \usepackage{hyperref}
    \usepackage{longtable} % longtable support required by pandoc >1.10
    \usepackage{booktabs}  % table support for pandoc > 1.12.2
    \usepackage[inline]{enumitem} % IRkernel/repr support (it uses the enumerate* environment)
    \usepackage[normalem]{ulem} % ulem is needed to support strikethroughs (\sout)
                                % normalem makes italics be italics, not underlines
    

    
    
    % Colors for the hyperref package
    \definecolor{urlcolor}{rgb}{0,.145,.698}
    \definecolor{linkcolor}{rgb}{.71,0.21,0.01}
    \definecolor{citecolor}{rgb}{.12,.54,.11}

    % ANSI colors
    \definecolor{ansi-black}{HTML}{3E424D}
    \definecolor{ansi-black-intense}{HTML}{282C36}
    \definecolor{ansi-red}{HTML}{E75C58}
    \definecolor{ansi-red-intense}{HTML}{B22B31}
    \definecolor{ansi-green}{HTML}{00A250}
    \definecolor{ansi-green-intense}{HTML}{007427}
    \definecolor{ansi-yellow}{HTML}{DDB62B}
    \definecolor{ansi-yellow-intense}{HTML}{B27D12}
    \definecolor{ansi-blue}{HTML}{208FFB}
    \definecolor{ansi-blue-intense}{HTML}{0065CA}
    \definecolor{ansi-magenta}{HTML}{D160C4}
    \definecolor{ansi-magenta-intense}{HTML}{A03196}
    \definecolor{ansi-cyan}{HTML}{60C6C8}
    \definecolor{ansi-cyan-intense}{HTML}{258F8F}
    \definecolor{ansi-white}{HTML}{C5C1B4}
    \definecolor{ansi-white-intense}{HTML}{A1A6B2}

    % commands and environments needed by pandoc snippets
    % extracted from the output of `pandoc -s`
    \providecommand{\tightlist}{%
      \setlength{\itemsep}{0pt}\setlength{\parskip}{0pt}}
    \DefineVerbatimEnvironment{Highlighting}{Verbatim}{commandchars=\\\{\}}
    % Add ',fontsize=\small' for more characters per line
    \newenvironment{Shaded}{}{}
    \newcommand{\KeywordTok}[1]{\textcolor[rgb]{0.00,0.44,0.13}{\textbf{{#1}}}}
    \newcommand{\DataTypeTok}[1]{\textcolor[rgb]{0.56,0.13,0.00}{{#1}}}
    \newcommand{\DecValTok}[1]{\textcolor[rgb]{0.25,0.63,0.44}{{#1}}}
    \newcommand{\BaseNTok}[1]{\textcolor[rgb]{0.25,0.63,0.44}{{#1}}}
    \newcommand{\FloatTok}[1]{\textcolor[rgb]{0.25,0.63,0.44}{{#1}}}
    \newcommand{\CharTok}[1]{\textcolor[rgb]{0.25,0.44,0.63}{{#1}}}
    \newcommand{\StringTok}[1]{\textcolor[rgb]{0.25,0.44,0.63}{{#1}}}
    \newcommand{\CommentTok}[1]{\textcolor[rgb]{0.38,0.63,0.69}{\textit{{#1}}}}
    \newcommand{\OtherTok}[1]{\textcolor[rgb]{0.00,0.44,0.13}{{#1}}}
    \newcommand{\AlertTok}[1]{\textcolor[rgb]{1.00,0.00,0.00}{\textbf{{#1}}}}
    \newcommand{\FunctionTok}[1]{\textcolor[rgb]{0.02,0.16,0.49}{{#1}}}
    \newcommand{\RegionMarkerTok}[1]{{#1}}
    \newcommand{\ErrorTok}[1]{\textcolor[rgb]{1.00,0.00,0.00}{\textbf{{#1}}}}
    \newcommand{\NormalTok}[1]{{#1}}
    
    % Additional commands for more recent versions of Pandoc
    \newcommand{\ConstantTok}[1]{\textcolor[rgb]{0.53,0.00,0.00}{{#1}}}
    \newcommand{\SpecialCharTok}[1]{\textcolor[rgb]{0.25,0.44,0.63}{{#1}}}
    \newcommand{\VerbatimStringTok}[1]{\textcolor[rgb]{0.25,0.44,0.63}{{#1}}}
    \newcommand{\SpecialStringTok}[1]{\textcolor[rgb]{0.73,0.40,0.53}{{#1}}}
    \newcommand{\ImportTok}[1]{{#1}}
    \newcommand{\DocumentationTok}[1]{\textcolor[rgb]{0.73,0.13,0.13}{\textit{{#1}}}}
    \newcommand{\AnnotationTok}[1]{\textcolor[rgb]{0.38,0.63,0.69}{\textbf{\textit{{#1}}}}}
    \newcommand{\CommentVarTok}[1]{\textcolor[rgb]{0.38,0.63,0.69}{\textbf{\textit{{#1}}}}}
    \newcommand{\VariableTok}[1]{\textcolor[rgb]{0.10,0.09,0.49}{{#1}}}
    \newcommand{\ControlFlowTok}[1]{\textcolor[rgb]{0.00,0.44,0.13}{\textbf{{#1}}}}
    \newcommand{\OperatorTok}[1]{\textcolor[rgb]{0.40,0.40,0.40}{{#1}}}
    \newcommand{\BuiltInTok}[1]{{#1}}
    \newcommand{\ExtensionTok}[1]{{#1}}
    \newcommand{\PreprocessorTok}[1]{\textcolor[rgb]{0.74,0.48,0.00}{{#1}}}
    \newcommand{\AttributeTok}[1]{\textcolor[rgb]{0.49,0.56,0.16}{{#1}}}
    \newcommand{\InformationTok}[1]{\textcolor[rgb]{0.38,0.63,0.69}{\textbf{\textit{{#1}}}}}
    \newcommand{\WarningTok}[1]{\textcolor[rgb]{0.38,0.63,0.69}{\textbf{\textit{{#1}}}}}
    
    
    % Define a nice break command that doesn't care if a line doesn't already
    % exist.
    \def\br{\hspace*{\fill} \\* }
    % Math Jax compatability definitions
    \def\gt{>}
    \def\lt{<}
    % Document parameters
    \title{Taller3 - TALLER DE PROGRAMACI?N NO LINEAL}
    
    
    

    % Pygments definitions
    
\makeatletter
\def\PY@reset{\let\PY@it=\relax \let\PY@bf=\relax%
    \let\PY@ul=\relax \let\PY@tc=\relax%
    \let\PY@bc=\relax \let\PY@ff=\relax}
\def\PY@tok#1{\csname PY@tok@#1\endcsname}
\def\PY@toks#1+{\ifx\relax#1\empty\else%
    \PY@tok{#1}\expandafter\PY@toks\fi}
\def\PY@do#1{\PY@bc{\PY@tc{\PY@ul{%
    \PY@it{\PY@bf{\PY@ff{#1}}}}}}}
\def\PY#1#2{\PY@reset\PY@toks#1+\relax+\PY@do{#2}}

\expandafter\def\csname PY@tok@w\endcsname{\def\PY@tc##1{\textcolor[rgb]{0.73,0.73,0.73}{##1}}}
\expandafter\def\csname PY@tok@c\endcsname{\let\PY@it=\textit\def\PY@tc##1{\textcolor[rgb]{0.25,0.50,0.50}{##1}}}
\expandafter\def\csname PY@tok@cp\endcsname{\def\PY@tc##1{\textcolor[rgb]{0.74,0.48,0.00}{##1}}}
\expandafter\def\csname PY@tok@k\endcsname{\let\PY@bf=\textbf\def\PY@tc##1{\textcolor[rgb]{0.00,0.50,0.00}{##1}}}
\expandafter\def\csname PY@tok@kp\endcsname{\def\PY@tc##1{\textcolor[rgb]{0.00,0.50,0.00}{##1}}}
\expandafter\def\csname PY@tok@kt\endcsname{\def\PY@tc##1{\textcolor[rgb]{0.69,0.00,0.25}{##1}}}
\expandafter\def\csname PY@tok@o\endcsname{\def\PY@tc##1{\textcolor[rgb]{0.40,0.40,0.40}{##1}}}
\expandafter\def\csname PY@tok@ow\endcsname{\let\PY@bf=\textbf\def\PY@tc##1{\textcolor[rgb]{0.67,0.13,1.00}{##1}}}
\expandafter\def\csname PY@tok@nb\endcsname{\def\PY@tc##1{\textcolor[rgb]{0.00,0.50,0.00}{##1}}}
\expandafter\def\csname PY@tok@nf\endcsname{\def\PY@tc##1{\textcolor[rgb]{0.00,0.00,1.00}{##1}}}
\expandafter\def\csname PY@tok@nc\endcsname{\let\PY@bf=\textbf\def\PY@tc##1{\textcolor[rgb]{0.00,0.00,1.00}{##1}}}
\expandafter\def\csname PY@tok@nn\endcsname{\let\PY@bf=\textbf\def\PY@tc##1{\textcolor[rgb]{0.00,0.00,1.00}{##1}}}
\expandafter\def\csname PY@tok@ne\endcsname{\let\PY@bf=\textbf\def\PY@tc##1{\textcolor[rgb]{0.82,0.25,0.23}{##1}}}
\expandafter\def\csname PY@tok@nv\endcsname{\def\PY@tc##1{\textcolor[rgb]{0.10,0.09,0.49}{##1}}}
\expandafter\def\csname PY@tok@no\endcsname{\def\PY@tc##1{\textcolor[rgb]{0.53,0.00,0.00}{##1}}}
\expandafter\def\csname PY@tok@nl\endcsname{\def\PY@tc##1{\textcolor[rgb]{0.63,0.63,0.00}{##1}}}
\expandafter\def\csname PY@tok@ni\endcsname{\let\PY@bf=\textbf\def\PY@tc##1{\textcolor[rgb]{0.60,0.60,0.60}{##1}}}
\expandafter\def\csname PY@tok@na\endcsname{\def\PY@tc##1{\textcolor[rgb]{0.49,0.56,0.16}{##1}}}
\expandafter\def\csname PY@tok@nt\endcsname{\let\PY@bf=\textbf\def\PY@tc##1{\textcolor[rgb]{0.00,0.50,0.00}{##1}}}
\expandafter\def\csname PY@tok@nd\endcsname{\def\PY@tc##1{\textcolor[rgb]{0.67,0.13,1.00}{##1}}}
\expandafter\def\csname PY@tok@s\endcsname{\def\PY@tc##1{\textcolor[rgb]{0.73,0.13,0.13}{##1}}}
\expandafter\def\csname PY@tok@sd\endcsname{\let\PY@it=\textit\def\PY@tc##1{\textcolor[rgb]{0.73,0.13,0.13}{##1}}}
\expandafter\def\csname PY@tok@si\endcsname{\let\PY@bf=\textbf\def\PY@tc##1{\textcolor[rgb]{0.73,0.40,0.53}{##1}}}
\expandafter\def\csname PY@tok@se\endcsname{\let\PY@bf=\textbf\def\PY@tc##1{\textcolor[rgb]{0.73,0.40,0.13}{##1}}}
\expandafter\def\csname PY@tok@sr\endcsname{\def\PY@tc##1{\textcolor[rgb]{0.73,0.40,0.53}{##1}}}
\expandafter\def\csname PY@tok@ss\endcsname{\def\PY@tc##1{\textcolor[rgb]{0.10,0.09,0.49}{##1}}}
\expandafter\def\csname PY@tok@sx\endcsname{\def\PY@tc##1{\textcolor[rgb]{0.00,0.50,0.00}{##1}}}
\expandafter\def\csname PY@tok@m\endcsname{\def\PY@tc##1{\textcolor[rgb]{0.40,0.40,0.40}{##1}}}
\expandafter\def\csname PY@tok@gh\endcsname{\let\PY@bf=\textbf\def\PY@tc##1{\textcolor[rgb]{0.00,0.00,0.50}{##1}}}
\expandafter\def\csname PY@tok@gu\endcsname{\let\PY@bf=\textbf\def\PY@tc##1{\textcolor[rgb]{0.50,0.00,0.50}{##1}}}
\expandafter\def\csname PY@tok@gd\endcsname{\def\PY@tc##1{\textcolor[rgb]{0.63,0.00,0.00}{##1}}}
\expandafter\def\csname PY@tok@gi\endcsname{\def\PY@tc##1{\textcolor[rgb]{0.00,0.63,0.00}{##1}}}
\expandafter\def\csname PY@tok@gr\endcsname{\def\PY@tc##1{\textcolor[rgb]{1.00,0.00,0.00}{##1}}}
\expandafter\def\csname PY@tok@ge\endcsname{\let\PY@it=\textit}
\expandafter\def\csname PY@tok@gs\endcsname{\let\PY@bf=\textbf}
\expandafter\def\csname PY@tok@gp\endcsname{\let\PY@bf=\textbf\def\PY@tc##1{\textcolor[rgb]{0.00,0.00,0.50}{##1}}}
\expandafter\def\csname PY@tok@go\endcsname{\def\PY@tc##1{\textcolor[rgb]{0.53,0.53,0.53}{##1}}}
\expandafter\def\csname PY@tok@gt\endcsname{\def\PY@tc##1{\textcolor[rgb]{0.00,0.27,0.87}{##1}}}
\expandafter\def\csname PY@tok@err\endcsname{\def\PY@bc##1{\setlength{\fboxsep}{0pt}\fcolorbox[rgb]{1.00,0.00,0.00}{1,1,1}{\strut ##1}}}
\expandafter\def\csname PY@tok@kc\endcsname{\let\PY@bf=\textbf\def\PY@tc##1{\textcolor[rgb]{0.00,0.50,0.00}{##1}}}
\expandafter\def\csname PY@tok@kd\endcsname{\let\PY@bf=\textbf\def\PY@tc##1{\textcolor[rgb]{0.00,0.50,0.00}{##1}}}
\expandafter\def\csname PY@tok@kn\endcsname{\let\PY@bf=\textbf\def\PY@tc##1{\textcolor[rgb]{0.00,0.50,0.00}{##1}}}
\expandafter\def\csname PY@tok@kr\endcsname{\let\PY@bf=\textbf\def\PY@tc##1{\textcolor[rgb]{0.00,0.50,0.00}{##1}}}
\expandafter\def\csname PY@tok@bp\endcsname{\def\PY@tc##1{\textcolor[rgb]{0.00,0.50,0.00}{##1}}}
\expandafter\def\csname PY@tok@fm\endcsname{\def\PY@tc##1{\textcolor[rgb]{0.00,0.00,1.00}{##1}}}
\expandafter\def\csname PY@tok@vc\endcsname{\def\PY@tc##1{\textcolor[rgb]{0.10,0.09,0.49}{##1}}}
\expandafter\def\csname PY@tok@vg\endcsname{\def\PY@tc##1{\textcolor[rgb]{0.10,0.09,0.49}{##1}}}
\expandafter\def\csname PY@tok@vi\endcsname{\def\PY@tc##1{\textcolor[rgb]{0.10,0.09,0.49}{##1}}}
\expandafter\def\csname PY@tok@vm\endcsname{\def\PY@tc##1{\textcolor[rgb]{0.10,0.09,0.49}{##1}}}
\expandafter\def\csname PY@tok@sa\endcsname{\def\PY@tc##1{\textcolor[rgb]{0.73,0.13,0.13}{##1}}}
\expandafter\def\csname PY@tok@sb\endcsname{\def\PY@tc##1{\textcolor[rgb]{0.73,0.13,0.13}{##1}}}
\expandafter\def\csname PY@tok@sc\endcsname{\def\PY@tc##1{\textcolor[rgb]{0.73,0.13,0.13}{##1}}}
\expandafter\def\csname PY@tok@dl\endcsname{\def\PY@tc##1{\textcolor[rgb]{0.73,0.13,0.13}{##1}}}
\expandafter\def\csname PY@tok@s2\endcsname{\def\PY@tc##1{\textcolor[rgb]{0.73,0.13,0.13}{##1}}}
\expandafter\def\csname PY@tok@sh\endcsname{\def\PY@tc##1{\textcolor[rgb]{0.73,0.13,0.13}{##1}}}
\expandafter\def\csname PY@tok@s1\endcsname{\def\PY@tc##1{\textcolor[rgb]{0.73,0.13,0.13}{##1}}}
\expandafter\def\csname PY@tok@mb\endcsname{\def\PY@tc##1{\textcolor[rgb]{0.40,0.40,0.40}{##1}}}
\expandafter\def\csname PY@tok@mf\endcsname{\def\PY@tc##1{\textcolor[rgb]{0.40,0.40,0.40}{##1}}}
\expandafter\def\csname PY@tok@mh\endcsname{\def\PY@tc##1{\textcolor[rgb]{0.40,0.40,0.40}{##1}}}
\expandafter\def\csname PY@tok@mi\endcsname{\def\PY@tc##1{\textcolor[rgb]{0.40,0.40,0.40}{##1}}}
\expandafter\def\csname PY@tok@il\endcsname{\def\PY@tc##1{\textcolor[rgb]{0.40,0.40,0.40}{##1}}}
\expandafter\def\csname PY@tok@mo\endcsname{\def\PY@tc##1{\textcolor[rgb]{0.40,0.40,0.40}{##1}}}
\expandafter\def\csname PY@tok@ch\endcsname{\let\PY@it=\textit\def\PY@tc##1{\textcolor[rgb]{0.25,0.50,0.50}{##1}}}
\expandafter\def\csname PY@tok@cm\endcsname{\let\PY@it=\textit\def\PY@tc##1{\textcolor[rgb]{0.25,0.50,0.50}{##1}}}
\expandafter\def\csname PY@tok@cpf\endcsname{\let\PY@it=\textit\def\PY@tc##1{\textcolor[rgb]{0.25,0.50,0.50}{##1}}}
\expandafter\def\csname PY@tok@c1\endcsname{\let\PY@it=\textit\def\PY@tc##1{\textcolor[rgb]{0.25,0.50,0.50}{##1}}}
\expandafter\def\csname PY@tok@cs\endcsname{\let\PY@it=\textit\def\PY@tc##1{\textcolor[rgb]{0.25,0.50,0.50}{##1}}}

\def\PYZbs{\char`\\}
\def\PYZus{\char`\_}
\def\PYZob{\char`\{}
\def\PYZcb{\char`\}}
\def\PYZca{\char`\^}
\def\PYZam{\char`\&}
\def\PYZlt{\char`\<}
\def\PYZgt{\char`\>}
\def\PYZsh{\char`\#}
\def\PYZpc{\char`\%}
\def\PYZdl{\char`\$}
\def\PYZhy{\char`\-}
\def\PYZsq{\char`\'}
\def\PYZdq{\char`\"}
\def\PYZti{\char`\~}
% for compatibility with earlier versions
\def\PYZat{@}
\def\PYZlb{[}
\def\PYZrb{]}
\makeatother


    % Exact colors from NB
    \definecolor{incolor}{rgb}{0.0, 0.0, 0.5}
    \definecolor{outcolor}{rgb}{0.545, 0.0, 0.0}



    
    % Prevent overflowing lines due to hard-to-break entities
    \sloppy 
    % Setup hyperref package
    \hypersetup{
      breaklinks=true,  % so long urls are correctly broken across lines
      colorlinks=true,
      urlcolor=urlcolor,
      linkcolor=linkcolor,
      citecolor=citecolor,
      }
    % Slightly bigger margins than the latex defaults
    
    \geometry{verbose,tmargin=1in,bmargin=1in,lmargin=1in,rmargin=1in}
    
    

    \begin{document}
    
    
    \maketitle
    
    

    
    \section{TALLER DE PROGRAMACIÓN NO
LINEAL}\label{taller-de-programaciuxf3n-no-lineal}

\subsubsection{Daniel Diaz - Alejandro
Suarez}\label{daniel-diaz---alejandro-suarez}

    \begin{Verbatim}[commandchars=\\\{\}]
{\color{incolor}In [{\color{incolor}1}]:} \PY{k+kn}{import} \PY{n+nn}{scipy}\PY{n+nn}{.}\PY{n+nn}{optimize} \PY{k}{as} \PY{n+nn}{optimize}
        \PY{k+kn}{import} \PY{n+nn}{numpy} \PY{k}{as} \PY{n+nn}{np}
\end{Verbatim}


    \subsubsection{1- Criterio:}\label{criterio}

f′+,f′′+=concave

f′+,f′′−=convex

f′−,f′′+=concave

f′−,f′′−=convex

\href{https://math.stackexchange.com/questions/852664/show-that-the-ellipse-and-the-hyperbola-are-convex}{the-ellipse-and-the-hyperbola-are-convex}

\paragraph{\texorpdfstring{fx=x1x2-x1\textsuperscript{2-x2}2
(Cóncava)}{fx=x1x2-x12-x22 (Cóncava)}}\label{fxx1x2-x12-x22-cuxf3ncava}

f'(x1) = x2-2x1 f''(x1) = -2 f''(x2, x1) = 1.

f'(x2) = x1-2x2 f''(x2) = -2 f''(x1, x2) = 1.

H= (H no es semidefinida positiva)

\begin{longtable}[]{@{}ll@{}}
\toprule
-2 & 1\tabularnewline
\midrule
\endhead
1 & -2\tabularnewline
\bottomrule
\end{longtable}

-H, det H = 5 Pero -H es semi definida positiva.

\paragraph{\texorpdfstring{fx=3x1+2x\textsuperscript{2+4x2+x2}2-2x1x2
(Convexa)}{fx=3x1+2x2+4x2+x22-2x1x2 (Convexa)}}\label{fx3x12x24x2x22-2x1x2-convexa}

f'(x1) = 3 + 2x1 - 2x2 f''(x1) = 2 f''(x2, x1) = -2.

f'(x2) = 4 + 2x2 - 2x1 f''(x2) = 2 f''(x1, x2) = -2.

\begin{longtable}[]{@{}ll@{}}
\toprule
2 & -2\tabularnewline
\midrule
\endhead
-2 & 2\tabularnewline
\bottomrule
\end{longtable}

H.detH = 8. Es semidefinida positiva.

\paragraph{\texorpdfstring{fx=x1\textsuperscript{2+2x2}2+3x1x2 (No es
cóncava, tampoco es
convexa)}{fx=x12+2x22+3x1x2 (No es cóncava, tampoco es convexa)}}\label{fxx122x223x1x2-no-es-cuxf3ncava-tampoco-es-convexa}

f'(x1) = 2x1 + 3x2 f''(x1) = 2 f''(x2, x1) = 3.

f'(x2) = 4x f''(x2) = 4 f''(x1, x2) = 3 .

H:

\begin{longtable}[]{@{}ll@{}}
\toprule
2 & 3\tabularnewline
\midrule
\endhead
3 & 4\tabularnewline
\bottomrule
\end{longtable}

det H = -1 No es semidefinida positiva. -H no es semidefinida positiva

\paragraph{fx=20x1+10x2 (convexa)}\label{fx20x110x2-convexa}

f'(x1) = 20 f''(x1) = 0 f''(x2, x1) = 0

f'(x2) = 10 f''(x2) = 0 f''(x1, x2) = 0

H:

\begin{longtable}[]{@{}ll@{}}
\toprule
0 & 0\tabularnewline
\midrule
\endhead
0 & 0\tabularnewline
\bottomrule
\end{longtable}

det H = 0. Es semidefinida positiva.

\paragraph{fx=x1+x2 (convexa)}\label{fxx1x2-convexa}

f'(x1) = x2 f''(x1) = 1 f''(x2, x1) = 0

f'(x2) = x1 f''(x2) = 1 f''(x1, x2) = 0

H:

\begin{longtable}[]{@{}ll@{}}
\toprule
1 & 0\tabularnewline
\midrule
\endhead
0 & 1\tabularnewline
\bottomrule
\end{longtable}

det H = 0. Es semidefinida positiva.

 

    \subsubsection{2- Aplicar Método de Bisección y
Newton:}\label{aplicar-muxe9todo-de-bisecciuxf3n-y-newton}

fx=x3+2x-2x\textsuperscript{2-0.25x}4

0≤x1≤24

    \begin{Verbatim}[commandchars=\\\{\}]
{\color{incolor}In [{\color{incolor}2}]:} \PY{k}{def} \PY{n+nf}{func1}\PY{p}{(}\PY{n}{x}\PY{p}{)}\PY{p}{:}
            \PY{k}{return} \PY{p}{(}\PY{n}{x}\PY{o}{*}\PY{o}{*}\PY{l+m+mi}{3} \PY{o}{+} \PY{l+m+mi}{2} \PY{o}{*} \PY{n}{x}\PY{p}{)} \PY{o}{\PYZhy{}} \PY{p}{(}\PY{l+m+mi}{2} \PY{o}{*} \PY{n}{x}\PY{o}{*}\PY{o}{*}\PY{l+m+mi}{2}\PY{p}{)} \PY{o}{\PYZhy{}} \PY{p}{(}\PY{l+m+mf}{0.25} \PY{o}{*} \PY{n}{x}\PY{o}{*}\PY{o}{*}\PY{l+m+mi}{4}\PY{p}{)}
        
        \PY{n+nb}{print}\PY{p}{(}\PY{l+s+s2}{\PYZdq{}}\PY{l+s+s2}{Bisection method:}\PY{l+s+s2}{\PYZdq{}}\PY{p}{,} \PY{n}{optimize}\PY{o}{.}\PY{n}{bisect}\PY{p}{(}\PY{n}{func1}\PY{p}{,} \PY{l+m+mi}{0}\PY{p}{,} \PY{l+m+mi}{24}\PY{p}{,} \PY{n}{full\PYZus{}output}\PY{o}{=}\PY{k+kc}{True}\PY{p}{,} \PY{n}{xtol}\PY{o}{=}\PY{l+m+mf}{0.004}\PY{p}{)}\PY{p}{)}
\end{Verbatim}


    \begin{Verbatim}[commandchars=\\\{\}]
Bisection method: (0.0,       converged: True
           flag: 'converged'
 function\_calls: 2
     iterations: 0
           root: 0.0)

    \end{Verbatim}

    \begin{Verbatim}[commandchars=\\\{\}]
{\color{incolor}In [{\color{incolor}3}]:} \PY{n+nb}{print}\PY{p}{(}\PY{l+s+s2}{\PYZdq{}}\PY{l+s+s2}{Newton method:}\PY{l+s+s2}{\PYZdq{}}\PY{p}{,} \PY{n}{optimize}\PY{o}{.}\PY{n}{newton}\PY{p}{(}\PY{n}{func1}\PY{p}{,} \PY{n}{x0}\PY{o}{=}\PY{l+m+mf}{1.2}\PY{p}{,} \PY{n}{tol}\PY{o}{=}\PY{l+m+mf}{0.001}\PY{p}{)}\PY{p}{)}
\end{Verbatim}


    \begin{Verbatim}[commandchars=\\\{\}]
Newton method: 2.000004219276435

    \end{Verbatim}

    \subsubsection{3- Aplicar Método de coordenadas cíclicas, gradiente y
Newton:}\label{aplicar-muxe9todo-de-coordenadas-cuxedclicas-gradiente-y-newton}

f(x1, x2) = 100(x2 - x1\textsuperscript{2)}2 + (1 - x1\^{}2) \#
rosenbrock function (multivariada = 2)

Se reemplaza Coordenadas cíclicas por nelder-mead el cual usa simplex
(el primero no se encuentra en el solver usado)

    \begin{Verbatim}[commandchars=\\\{\}]
{\color{incolor}In [{\color{incolor}4}]:} \PY{k}{def} \PY{n+nf}{rosen}\PY{p}{(}\PY{n}{x}\PY{p}{)}\PY{p}{:}
            \PY{l+s+sd}{\PYZdq{}\PYZdq{}\PYZdq{}The Rosenbrock function\PYZdq{}\PYZdq{}\PYZdq{}}
            \PY{k}{return} \PY{n+nb}{sum}\PY{p}{(}\PY{l+m+mf}{100.0}\PY{o}{*}\PY{p}{(}\PY{n}{x}\PY{p}{[}\PY{l+m+mi}{1}\PY{p}{:}\PY{p}{]}\PY{o}{\PYZhy{}}\PY{n}{x}\PY{p}{[}\PY{p}{:}\PY{o}{\PYZhy{}}\PY{l+m+mi}{1}\PY{p}{]}\PY{o}{*}\PY{o}{*}\PY{l+m+mf}{2.0}\PY{p}{)}\PY{o}{*}\PY{o}{*}\PY{l+m+mf}{2.0} \PY{o}{+} \PY{p}{(}\PY{l+m+mi}{1}\PY{o}{\PYZhy{}}\PY{n}{x}\PY{p}{[}\PY{p}{:}\PY{o}{\PYZhy{}}\PY{l+m+mi}{1}\PY{p}{]}\PY{p}{)}\PY{o}{*}\PY{o}{*}\PY{l+m+mf}{2.0}\PY{p}{)}
        
        \PY{k}{def} \PY{n+nf}{rosen\PYZus{}der}\PY{p}{(}\PY{n}{x}\PY{p}{)}\PY{p}{:} \PY{c+c1}{\PYZsh{} The gradient of the Rosenbrock function is the vector:}
            \PY{n}{xm} \PY{o}{=} \PY{n}{x}\PY{p}{[}\PY{l+m+mi}{1}\PY{p}{:}\PY{o}{\PYZhy{}}\PY{l+m+mi}{1}\PY{p}{]}
            \PY{n}{xm\PYZus{}m1} \PY{o}{=} \PY{n}{x}\PY{p}{[}\PY{p}{:}\PY{o}{\PYZhy{}}\PY{l+m+mi}{2}\PY{p}{]}
            \PY{n}{xm\PYZus{}p1} \PY{o}{=} \PY{n}{x}\PY{p}{[}\PY{l+m+mi}{2}\PY{p}{:}\PY{p}{]}
            \PY{n}{der} \PY{o}{=} \PY{n}{np}\PY{o}{.}\PY{n}{zeros\PYZus{}like}\PY{p}{(}\PY{n}{x}\PY{p}{)}
            \PY{n}{der}\PY{p}{[}\PY{l+m+mi}{1}\PY{p}{:}\PY{o}{\PYZhy{}}\PY{l+m+mi}{1}\PY{p}{]} \PY{o}{=} \PY{l+m+mi}{200}\PY{o}{*}\PY{p}{(}\PY{n}{xm}\PY{o}{\PYZhy{}}\PY{n}{xm\PYZus{}m1}\PY{o}{*}\PY{o}{*}\PY{l+m+mi}{2}\PY{p}{)} \PY{o}{\PYZhy{}} \PY{l+m+mi}{400}\PY{o}{*}\PY{p}{(}\PY{n}{xm\PYZus{}p1} \PY{o}{\PYZhy{}} \PY{n}{xm}\PY{o}{*}\PY{o}{*}\PY{l+m+mi}{2}\PY{p}{)}\PY{o}{*}\PY{n}{xm} \PY{o}{\PYZhy{}} \PY{l+m+mi}{2}\PY{o}{*}\PY{p}{(}\PY{l+m+mi}{1}\PY{o}{\PYZhy{}}\PY{n}{xm}\PY{p}{)}
            \PY{n}{der}\PY{p}{[}\PY{l+m+mi}{0}\PY{p}{]} \PY{o}{=} \PY{o}{\PYZhy{}}\PY{l+m+mi}{400}\PY{o}{*}\PY{n}{x}\PY{p}{[}\PY{l+m+mi}{0}\PY{p}{]}\PY{o}{*}\PY{p}{(}\PY{n}{x}\PY{p}{[}\PY{l+m+mi}{1}\PY{p}{]}\PY{o}{\PYZhy{}}\PY{n}{x}\PY{p}{[}\PY{l+m+mi}{0}\PY{p}{]}\PY{o}{*}\PY{o}{*}\PY{l+m+mi}{2}\PY{p}{)} \PY{o}{\PYZhy{}} \PY{l+m+mi}{2}\PY{o}{*}\PY{p}{(}\PY{l+m+mi}{1}\PY{o}{\PYZhy{}}\PY{n}{x}\PY{p}{[}\PY{l+m+mi}{0}\PY{p}{]}\PY{p}{)}
            \PY{n}{der}\PY{p}{[}\PY{o}{\PYZhy{}}\PY{l+m+mi}{1}\PY{p}{]} \PY{o}{=} \PY{l+m+mi}{200}\PY{o}{*}\PY{p}{(}\PY{n}{x}\PY{p}{[}\PY{o}{\PYZhy{}}\PY{l+m+mi}{1}\PY{p}{]}\PY{o}{\PYZhy{}}\PY{n}{x}\PY{p}{[}\PY{o}{\PYZhy{}}\PY{l+m+mi}{2}\PY{p}{]}\PY{o}{*}\PY{o}{*}\PY{l+m+mi}{2}\PY{p}{)}
            \PY{k}{return} \PY{n}{der}
        
        \PY{k}{def} \PY{n+nf}{rosen\PYZus{}hess}\PY{p}{(}\PY{n}{x}\PY{p}{)}\PY{p}{:}
            \PY{n}{x} \PY{o}{=} \PY{n}{np}\PY{o}{.}\PY{n}{asarray}\PY{p}{(}\PY{n}{x}\PY{p}{)}
            \PY{n}{H} \PY{o}{=} \PY{n}{np}\PY{o}{.}\PY{n}{diag}\PY{p}{(}\PY{o}{\PYZhy{}}\PY{l+m+mi}{400}\PY{o}{*}\PY{n}{x}\PY{p}{[}\PY{p}{:}\PY{o}{\PYZhy{}}\PY{l+m+mi}{1}\PY{p}{]}\PY{p}{,}\PY{l+m+mi}{1}\PY{p}{)} \PY{o}{\PYZhy{}} \PY{n}{np}\PY{o}{.}\PY{n}{diag}\PY{p}{(}\PY{l+m+mi}{400}\PY{o}{*}\PY{n}{x}\PY{p}{[}\PY{p}{:}\PY{o}{\PYZhy{}}\PY{l+m+mi}{1}\PY{p}{]}\PY{p}{,}\PY{o}{\PYZhy{}}\PY{l+m+mi}{1}\PY{p}{)}
            \PY{n}{diagonal} \PY{o}{=} \PY{n}{np}\PY{o}{.}\PY{n}{zeros\PYZus{}like}\PY{p}{(}\PY{n}{x}\PY{p}{)}
            \PY{n}{diagonal}\PY{p}{[}\PY{l+m+mi}{0}\PY{p}{]} \PY{o}{=} \PY{l+m+mi}{1200}\PY{o}{*}\PY{n}{x}\PY{p}{[}\PY{l+m+mi}{0}\PY{p}{]}\PY{o}{*}\PY{o}{*}\PY{l+m+mi}{2}\PY{o}{\PYZhy{}}\PY{l+m+mi}{400}\PY{o}{*}\PY{n}{x}\PY{p}{[}\PY{l+m+mi}{1}\PY{p}{]}\PY{o}{+}\PY{l+m+mi}{2}
            \PY{n}{diagonal}\PY{p}{[}\PY{o}{\PYZhy{}}\PY{l+m+mi}{1}\PY{p}{]} \PY{o}{=} \PY{l+m+mi}{200}
            \PY{n}{diagonal}\PY{p}{[}\PY{l+m+mi}{1}\PY{p}{:}\PY{o}{\PYZhy{}}\PY{l+m+mi}{1}\PY{p}{]} \PY{o}{=} \PY{l+m+mi}{202} \PY{o}{+} \PY{l+m+mi}{1200}\PY{o}{*}\PY{n}{x}\PY{p}{[}\PY{l+m+mi}{1}\PY{p}{:}\PY{o}{\PYZhy{}}\PY{l+m+mi}{1}\PY{p}{]}\PY{o}{*}\PY{o}{*}\PY{l+m+mi}{2} \PY{o}{\PYZhy{}} \PY{l+m+mi}{400}\PY{o}{*}\PY{n}{x}\PY{p}{[}\PY{l+m+mi}{2}\PY{p}{:}\PY{p}{]}
            \PY{n}{H} \PY{o}{=} \PY{n}{H} \PY{o}{+} \PY{n}{np}\PY{o}{.}\PY{n}{diag}\PY{p}{(}\PY{n}{diagonal}\PY{p}{)}
            \PY{k}{return} \PY{n}{H}
        
        \PY{n}{x0} \PY{o}{=} \PY{n}{np}\PY{o}{.}\PY{n}{array}\PY{p}{(}\PY{p}{[}\PY{l+m+mi}{0}\PY{p}{,} \PY{l+m+mi}{0}\PY{p}{]}\PY{p}{)}
        
        \PY{n+nb}{print}\PY{p}{(}\PY{l+s+s2}{\PYZdq{}}\PY{l+s+se}{\PYZbs{}n}\PY{l+s+s2}{Newton CG method:}\PY{l+s+s2}{\PYZdq{}}\PY{p}{,} \PY{n}{optimize}\PY{o}{.}\PY{n}{minimize}\PY{p}{(}\PY{n}{rosen}\PY{p}{,} \PY{n}{x0}\PY{p}{,} \PY{n}{method}\PY{o}{=}\PY{l+s+s1}{\PYZsq{}}\PY{l+s+s1}{Newton\PYZhy{}CG}\PY{l+s+s1}{\PYZsq{}}\PY{p}{,} \PY{n}{jac}\PY{o}{=}\PY{n}{rosen\PYZus{}der}\PY{p}{,} \PY{n}{hess}\PY{o}{=}\PY{n}{rosen\PYZus{}hess}\PY{p}{,} \PY{n}{options}\PY{o}{=}\PY{p}{\PYZob{}}\PY{l+s+s1}{\PYZsq{}}\PY{l+s+s1}{xtol}\PY{l+s+s1}{\PYZsq{}}\PY{p}{:} \PY{l+m+mf}{1e\PYZhy{}8}\PY{p}{,} \PY{l+s+s1}{\PYZsq{}}\PY{l+s+s1}{disp}\PY{l+s+s1}{\PYZsq{}}\PY{p}{:} \PY{k+kc}{True}\PY{p}{\PYZcb{}}\PY{p}{)}\PY{p}{)}
\end{Verbatim}


    \begin{Verbatim}[commandchars=\\\{\}]
Optimization terminated successfully.
         Current function value: 0.000000
         Iterations: 35
         Function evaluations: 55
         Gradient evaluations: 89
         Hessian evaluations: 35

Newton CG method:      fun: 3.2093766919136996e-18
     jac: array([ 5.94461863e-07, -2.98427461e-07])
 message: 'Optimization terminated successfully.'
    nfev: 55
    nhev: 35
     nit: 35
    njev: 89
  status: 0
 success: True
       x: array([1., 1.])

    \end{Verbatim}

    \begin{Verbatim}[commandchars=\\\{\}]
{\color{incolor}In [{\color{incolor}5}]:} \PY{n+nb}{print}\PY{p}{(}\PY{l+s+s2}{\PYZdq{}}\PY{l+s+s2}{Conjugate gradient descent}\PY{l+s+se}{\PYZbs{}n}\PY{l+s+s2}{\PYZdq{}}\PY{p}{,} \PY{n}{optimize}\PY{o}{.}\PY{n}{minimize}\PY{p}{(}\PY{n}{rosen}\PY{p}{,} \PY{p}{[}\PY{l+m+mi}{0}\PY{p}{,} \PY{l+m+mi}{0}\PY{p}{]}\PY{p}{,} \PY{n}{method}\PY{o}{=}\PY{l+s+s2}{\PYZdq{}}\PY{l+s+s2}{CG}\PY{l+s+s2}{\PYZdq{}}\PY{p}{)}\PY{p}{)}
\end{Verbatim}


    \begin{Verbatim}[commandchars=\\\{\}]
Conjugate gradient descent
      fun: 2.0085382242752512e-11
     jac: array([ 5.41245606e-06, -2.70523446e-06])
 message: 'Optimization terminated successfully.'
    nfev: 220
     nit: 21
    njev: 55
  status: 0
 success: True
       x: array([0.99999552, 0.99999103])

    \end{Verbatim}

    \begin{Verbatim}[commandchars=\\\{\}]
{\color{incolor}In [{\color{incolor}6}]:} \PY{n+nb}{print}\PY{p}{(}\PY{l+s+s2}{\PYZdq{}}\PY{l+s+s2}{Nelder\PYZhy{}Mead}\PY{l+s+se}{\PYZbs{}n}\PY{l+s+s2}{\PYZdq{}}\PY{p}{,} \PY{n}{optimize}\PY{o}{.}\PY{n}{minimize}\PY{p}{(}\PY{n}{rosen}\PY{p}{,} \PY{p}{[}\PY{l+m+mi}{0}\PY{p}{,} \PY{l+m+mi}{0}\PY{p}{]}\PY{p}{,} \PY{n}{method}\PY{o}{=}\PY{l+s+s1}{\PYZsq{}}\PY{l+s+s1}{nelder\PYZhy{}mead}\PY{l+s+s1}{\PYZsq{}}\PY{p}{,} \PY{n}{options}\PY{o}{=}\PY{p}{\PYZob{}}\PY{l+s+s1}{\PYZsq{}}\PY{l+s+s1}{xtol}\PY{l+s+s1}{\PYZsq{}}\PY{p}{:} \PY{l+m+mf}{1e\PYZhy{}8}\PY{p}{,} \PY{l+s+s1}{\PYZsq{}}\PY{l+s+s1}{disp}\PY{l+s+s1}{\PYZsq{}}\PY{p}{:} \PY{k+kc}{True}\PY{p}{\PYZcb{}}\PY{p}{)}\PY{p}{)}
\end{Verbatim}


    \begin{Verbatim}[commandchars=\\\{\}]
Optimization terminated successfully.
         Current function value: 0.000000
         Iterations: 110
         Function evaluations: 206
Nelder-Mead
  final\_simplex: (array([[1., 1.],
       [1., 1.],
       [1., 1.]]), array([2.65045900e-18, 6.43252125e-18, 1.09257777e-17]))
           fun: 2.650458998741964e-18
       message: 'Optimization terminated successfully.'
          nfev: 206
           nit: 110
        status: 0
       success: True
             x: array([1., 1.])

    \end{Verbatim}

    \subsubsection{4- problema de programación no
lineal:}\label{problema-de-programaciuxf3n-no-lineal}

z = (x1 - 9/4) \^{} 2 + (x2 - 2) \^{} 2

S.A

x2-x1\^{}2≥0

x1+x2≤6

x1,x2 ≥0

\paragraph{Naturaleza de la función:}\label{naturaleza-de-la-funciuxf3n}

\paragraph{condiciones de optimalidad}\label{condiciones-de-optimalidad}

Forma estándar:

S.A:

E1 = x1\^{}2 − x2 ≤ 0

E2 = x1 + x2 − 6 ≤ 0

E3 = − x1 ≤ 0

E4 = − x2 ≤ 0

\begin{center}\rule{0.5\linewidth}{\linethickness}\end{center}

∇f = (−2(x −2(x2 1 − − 9/4) (-2(x2 − 2))

∇E1 = (2x1) (x1)

∇E2 = (1) (1)

∇E3 = (-1) (0)

∇E4 = (0) (-1)

\begin{center}\rule{0.5\linewidth}{\linethickness}\end{center}

Asumiendo x* = (3/2) (9/4) las condiciones KKT:

∇f + μ1 ∇E1 + μ2 ∇E2 + μ3 ∇E3 + μ4 ∇E4 = 0

μ1 E1 = 0

μ2 E2 = − 27/4 μ2 = 0 ---\textgreater{} μ2 = 0

μ3 E3 = − 3/2 μ3 = 0 ---\textgreater{} μ3 = 0

μ4 E4 = − 9/4 μ4 = 0 ---\textgreater{} μ4 = 0

---\textgreater{} ∇f + μ1 ∇E1 = (1/2) (3/2)+ μ1 (−1) (3) = 0
---\textgreater{} μ1 = 1/2

Es un mínimo gloal, y es convexa.

    \begin{Verbatim}[commandchars=\\\{\}]
{\color{incolor}In [{\color{incolor}7}]:} \PY{k}{def} \PY{n+nf}{objective}\PY{p}{(}\PY{n}{x}\PY{p}{)}\PY{p}{:}
            \PY{k}{return} \PY{p}{(}\PY{p}{(}\PY{n}{x}\PY{p}{[}\PY{l+m+mi}{0}\PY{p}{]} \PY{o}{\PYZhy{}} \PY{l+m+mi}{9}\PY{o}{/}\PY{l+m+mi}{4}\PY{p}{)}\PY{o}{*}\PY{o}{*}\PY{l+m+mi}{2} \PY{o}{+} \PY{p}{(}\PY{n}{x}\PY{p}{[}\PY{l+m+mi}{1}\PY{p}{]} \PY{o}{\PYZhy{}} \PY{l+m+mi}{2}\PY{p}{)}\PY{o}{*}\PY{o}{*}\PY{l+m+mi}{2}\PY{p}{)}
        
        \PY{k}{def} \PY{n+nf}{constraint1}\PY{p}{(}\PY{n}{x}\PY{p}{)}\PY{p}{:}
            \PY{k}{return} \PY{n}{x}\PY{p}{[}\PY{l+m+mi}{1}\PY{p}{]} \PY{o}{\PYZhy{}} \PY{n}{x}\PY{p}{[}\PY{l+m+mi}{0}\PY{p}{]}\PY{o}{*}\PY{o}{*}\PY{l+m+mi}{2}
        
        \PY{k}{def} \PY{n+nf}{constraint2}\PY{p}{(}\PY{n}{x}\PY{p}{)}\PY{p}{:}
            \PY{k}{return} \PY{n}{x}\PY{p}{[}\PY{l+m+mi}{0}\PY{p}{]} \PY{o}{\PYZhy{}} \PY{n}{x}\PY{p}{[}\PY{l+m+mi}{1}\PY{p}{]} \PY{o}{+} \PY{l+m+mi}{6}
        
        
        \PY{n}{x0} \PY{o}{=} \PY{n}{np}\PY{o}{.}\PY{n}{zeros}\PY{p}{(}\PY{l+m+mi}{2}\PY{p}{)}
        
        \PY{c+c1}{\PYZsh{} show initial objective}
        \PY{n+nb}{print}\PY{p}{(}\PY{l+s+s1}{\PYZsq{}}\PY{l+s+s1}{Initial SSE Objective: }\PY{l+s+s1}{\PYZsq{}} \PY{o}{+} \PY{n+nb}{str}\PY{p}{(}\PY{n}{objective}\PY{p}{(}\PY{n}{x0}\PY{p}{)}\PY{p}{)}\PY{p}{)}
        
        \PY{c+c1}{\PYZsh{} optimize}
        
        \PY{n}{con1} \PY{o}{=} \PY{p}{\PYZob{}}\PY{l+s+s1}{\PYZsq{}}\PY{l+s+s1}{type}\PY{l+s+s1}{\PYZsq{}}\PY{p}{:} \PY{l+s+s1}{\PYZsq{}}\PY{l+s+s1}{ineq}\PY{l+s+s1}{\PYZsq{}}\PY{p}{,} \PY{l+s+s1}{\PYZsq{}}\PY{l+s+s1}{fun}\PY{l+s+s1}{\PYZsq{}}\PY{p}{:} \PY{n}{constraint1}\PY{p}{\PYZcb{}}
        \PY{n}{con2} \PY{o}{=} \PY{p}{\PYZob{}}\PY{l+s+s1}{\PYZsq{}}\PY{l+s+s1}{type}\PY{l+s+s1}{\PYZsq{}}\PY{p}{:} \PY{l+s+s1}{\PYZsq{}}\PY{l+s+s1}{ineq}\PY{l+s+s1}{\PYZsq{}}\PY{p}{,} \PY{l+s+s1}{\PYZsq{}}\PY{l+s+s1}{fun}\PY{l+s+s1}{\PYZsq{}}\PY{p}{:} \PY{n}{constraint2}\PY{p}{\PYZcb{}}
        \PY{n}{cons} \PY{o}{=} \PY{p}{(}\PY{p}{[}\PY{n}{con1}\PY{p}{,}\PY{n}{con2}\PY{p}{]}\PY{p}{)}
        
        \PY{n}{solution} \PY{o}{=} \PY{n}{optimize}\PY{o}{.}\PY{n}{minimize}\PY{p}{(}\PY{n}{objective}\PY{p}{,} \PY{n}{x0}\PY{p}{,} \PY{n}{method}\PY{o}{=}\PY{l+s+s1}{\PYZsq{}}\PY{l+s+s1}{SLSQP}\PY{l+s+s1}{\PYZsq{}}\PY{p}{,}\PY{n}{constraints}\PY{o}{=}\PY{n}{cons}\PY{p}{)}
        
        \PY{n}{x} \PY{o}{=} \PY{n}{solution}\PY{o}{.}\PY{n}{x}
        
        \PY{c+c1}{\PYZsh{} show final objective}
        \PY{n+nb}{print}\PY{p}{(}\PY{l+s+s1}{\PYZsq{}}\PY{l+s+s1}{Final SSE Objective: }\PY{l+s+s1}{\PYZsq{}} \PY{o}{+} \PY{n+nb}{str}\PY{p}{(}\PY{n}{objective}\PY{p}{(}\PY{n}{x}\PY{p}{)}\PY{p}{)}\PY{p}{)}
        
        \PY{c+c1}{\PYZsh{} print solution}
        \PY{n+nb}{print}\PY{p}{(}\PY{l+s+s1}{\PYZsq{}}\PY{l+s+s1}{Solution}\PY{l+s+s1}{\PYZsq{}}\PY{p}{)}
        \PY{n+nb}{print}\PY{p}{(}\PY{l+s+s1}{\PYZsq{}}\PY{l+s+s1}{x1 = }\PY{l+s+s1}{\PYZsq{}} \PY{o}{+} \PY{n+nb}{str}\PY{p}{(}\PY{n}{x}\PY{p}{[}\PY{l+m+mi}{0}\PY{p}{]}\PY{p}{)}\PY{p}{)}
        \PY{n+nb}{print}\PY{p}{(}\PY{l+s+s1}{\PYZsq{}}\PY{l+s+s1}{x2 = }\PY{l+s+s1}{\PYZsq{}} \PY{o}{+} \PY{n+nb}{str}\PY{p}{(}\PY{n}{x}\PY{p}{[}\PY{l+m+mi}{1}\PY{p}{]}\PY{p}{)}\PY{p}{)}
\end{Verbatim}


    \begin{Verbatim}[commandchars=\\\{\}]
Initial SSE Objective: 9.0625
Final SSE Objective: 0.6249999947299062
Solution
x1 = 1.500000011329049
x2 = 2.250000023446958

    \end{Verbatim}

    \subsubsection{5- problema de programación no lineal (optimos
extremos):}\label{problema-de-programaciuxf3n-no-lineal-optimos-extremos}

z = x1 + x2

S.A

x2\^{}2 - x1\^{}2≤1

x1,x2 ≥0

f'(x1) = 1, f''(x1) = 0, f''(x2, x1) = 0

f'(x2) = 1, f''(x2) = 0, f''(x1, x2) = 0

\begin{enumerate}
\def\labelenumi{\Alph{enumi})}
\tightlist
\item
  Puntos extremos
\end{enumerate}

f = − x1 − x2 E1 = x1\^{}2 + x2\^{}2 − 1 E2 = −x1 E3 = −x2

∇f = (-1 , 1) ∇E1 = (2x1, 2x2) ∇E2 = (-1, 0) ∇E3 = (0, -1)

\begin{center}\rule{0.5\linewidth}{\linethickness}\end{center}

KKT:

2x1 μ1 − μ2 = 1 -\textgreater{} μ2 = 2x1μ1 − 1

2x2μ1 − μ3 = 1 -\textgreater{} μ3 = 2x2μ1 − 1

μ1(x1\^{}2 + x2\^{}2 − 1) = 0 ---\textgreater{} μ1(x1\^{}2 + x2\^{}2 −
1) = 0

μ2x1 = 0 -\textgreater{} μ2x1 = 0 μ3x2 = 0-\textgreater{} μ3x2 = 0

\begin{center}\rule{0.5\linewidth}{\linethickness}\end{center}

x1 = 1 / 2μ1

x2 = 1 / 2μ1

± 1/2\^{}(-1/2)

\begin{center}\rule{0.5\linewidth}{\linethickness}\end{center}

\paragraph{condiciones de optimalidad}\label{condiciones-de-optimalidad}

x =

\begin{longtable}[]{@{}l@{}}
\toprule
0.71\tabularnewline
\midrule
\endhead
0.71\tabularnewline
\bottomrule
\end{longtable}

f(x) = 1.411 (óptimo)

\paragraph{Naturaleza de la función:}\label{naturaleza-de-la-funciuxf3n}

 

    \subsubsection{6- Considere el siguiente problema de programación lineal
fraccional}\label{considere-el-siguiente-problema-de-programaciuxf3n-lineal-fraccional}

f(x) = 10x1+20x2+10 / 3x1+4x2+20

s.a.

1x1+3x2≤50

3x1+2x2≤50

x1,x2≥0

\begin{center}\rule{0.5\linewidth}{\linethickness}\end{center}

Usando la transformación de Charnes-Cooper:

y = (1 / (3x1 + 4x2 + 20)) x

t = (1 / 3x1 + 4x2 + 20)

max z′ = 10y 1 + 20y 2 + 10t

S.A

y1 + 3y2 − 50t ≤ 0 3y1 + 2y2 − 50t ≤ 0 3y1 + 4y2 + 20t = 1

y1, y2, t ≥ 0

\begin{center}\rule{0.5\linewidth}{\linethickness}\end{center}

y1 + 3y2 − 50t + s1 + a1 = 0

3y1 + 2y2 − 50t + s2 + a2 = 0

3y1 + 4y2 + 20t + a3 = 1

\begin{center}\rule{0.5\linewidth}{\linethickness}\end{center}

Usando un solver de programación lineal: y = (0, 5/26)

t = (260/3)

x* = 1/t

y = (0, 50/3)

con z = 103/26

    https://www.geogebra.org/graphing

https://www.symbolab.com/solver/function-inflection-points-calculator/

https://www.scipy-lectures.org/advanced/mathematical\_optimization/

http://apmonitor.com/che263/index.php/Main/PythonOptimization

https://apmonitor.com/pdc/index.php/Main/NonlinearProgramming

https://www.geogebra.org/3d?lang=es


    % Add a bibliography block to the postdoc
    
    
    
    \end{document}
